\documentclass[11pt]{article}

    \usepackage[breakable]{tcolorbox}
    \usepackage{parskip} % Stop auto-indenting (to mimic markdown behaviour)
    

    % Basic figure setup, for now with no caption control since it's done
    % automatically by Pandoc (which extracts ![](path) syntax from Markdown).
    \usepackage{graphicx}
    % Maintain compatibility with old templates. Remove in nbconvert 6.0
    \let\Oldincludegraphics\includegraphics
    % Ensure that by default, figures have no caption (until we provide a
    % proper Figure object with a Caption API and a way to capture that
    % in the conversion process - todo).
    \usepackage{caption}
    \DeclareCaptionFormat{nocaption}{}
    \captionsetup{format=nocaption,aboveskip=0pt,belowskip=0pt}

    \usepackage{float}
    \floatplacement{figure}{H} % forces figures to be placed at the correct location
    \usepackage{xcolor} % Allow colors to be defined
    \usepackage{enumerate} % Needed for markdown enumerations to work
    \usepackage{geometry} % Used to adjust the document margins
    \usepackage{amsmath} % Equations
    \usepackage{amssymb} % Equations
    \usepackage{textcomp} % defines textquotesingle
    % Hack from http://tex.stackexchange.com/a/47451/13684:
    \AtBeginDocument{%
        \def\PYZsq{\textquotesingle}% Upright quotes in Pygmentized code
    }
    \usepackage{upquote} % Upright quotes for verbatim code
    \usepackage{eurosym} % defines \euro

    \usepackage{iftex}
    \ifPDFTeX
        \usepackage[T1]{fontenc}
        \IfFileExists{alphabeta.sty}{
              \usepackage{alphabeta}
          }{
              \usepackage[mathletters]{ucs}
              \usepackage[utf8x]{inputenc}
          }
    \else
        \usepackage{fontspec}
        \usepackage{unicode-math}
    \fi

    \usepackage{fancyvrb} % verbatim replacement that allows latex
    \usepackage{grffile} % extends the file name processing of package graphics
                         % to support a larger range
    \makeatletter % fix for old versions of grffile with XeLaTeX
    \@ifpackagelater{grffile}{2019/11/01}
    {
      % Do nothing on new versions
    }
    {
      \def\Gread@@xetex#1{%
        \IfFileExists{"\Gin@base".bb}%
        {\Gread@eps{\Gin@base.bb}}%
        {\Gread@@xetex@aux#1}%
      }
    }
    \makeatother
    \usepackage[Export]{adjustbox} % Used to constrain images to a maximum size
    \adjustboxset{max size={0.9\linewidth}{0.9\paperheight}}

    % The hyperref package gives us a pdf with properly built
    % internal navigation ('pdf bookmarks' for the table of contents,
    % internal cross-reference links, web links for URLs, etc.)
    \usepackage{hyperref}
    % The default LaTeX title has an obnoxious amount of whitespace. By default,
    % titling removes some of it. It also provides customization options.
    \usepackage{titling}
    \usepackage{longtable} % longtable support required by pandoc >1.10
    \usepackage{booktabs}  % table support for pandoc > 1.12.2
    \usepackage{array}     % table support for pandoc >= 2.11.3
    \usepackage{calc}      % table minipage width calculation for pandoc >= 2.11.1
    \usepackage[inline]{enumitem} % IRkernel/repr support (it uses the enumerate* environment)
    \usepackage[normalem]{ulem} % ulem is needed to support strikethroughs (\sout)
                                % normalem makes italics be italics, not underlines
    \usepackage{mathrsfs}
    

    
    % Colors for the hyperref package
    \definecolor{urlcolor}{rgb}{0,.145,.698}
    \definecolor{linkcolor}{rgb}{.71,0.21,0.01}
    \definecolor{citecolor}{rgb}{.12,.54,.11}

    % ANSI colors
    \definecolor{ansi-black}{HTML}{3E424D}
    \definecolor{ansi-black-intense}{HTML}{282C36}
    \definecolor{ansi-red}{HTML}{E75C58}
    \definecolor{ansi-red-intense}{HTML}{B22B31}
    \definecolor{ansi-green}{HTML}{00A250}
    \definecolor{ansi-green-intense}{HTML}{007427}
    \definecolor{ansi-yellow}{HTML}{DDB62B}
    \definecolor{ansi-yellow-intense}{HTML}{B27D12}
    \definecolor{ansi-blue}{HTML}{208FFB}
    \definecolor{ansi-blue-intense}{HTML}{0065CA}
    \definecolor{ansi-magenta}{HTML}{D160C4}
    \definecolor{ansi-magenta-intense}{HTML}{A03196}
    \definecolor{ansi-cyan}{HTML}{60C6C8}
    \definecolor{ansi-cyan-intense}{HTML}{258F8F}
    \definecolor{ansi-white}{HTML}{C5C1B4}
    \definecolor{ansi-white-intense}{HTML}{A1A6B2}
    \definecolor{ansi-default-inverse-fg}{HTML}{FFFFFF}
    \definecolor{ansi-default-inverse-bg}{HTML}{000000}

    % common color for the border for error outputs.
    \definecolor{outerrorbackground}{HTML}{FFDFDF}

    % commands and environments needed by pandoc snippets
    % extracted from the output of `pandoc -s`
    \providecommand{\tightlist}{%
      \setlength{\itemsep}{0pt}\setlength{\parskip}{0pt}}
    \DefineVerbatimEnvironment{Highlighting}{Verbatim}{commandchars=\\\{\}}
    % Add ',fontsize=\small' for more characters per line
    \newenvironment{Shaded}{}{}
    \newcommand{\KeywordTok}[1]{\textcolor[rgb]{0.00,0.44,0.13}{\textbf{{#1}}}}
    \newcommand{\DataTypeTok}[1]{\textcolor[rgb]{0.56,0.13,0.00}{{#1}}}
    \newcommand{\DecValTok}[1]{\textcolor[rgb]{0.25,0.63,0.44}{{#1}}}
    \newcommand{\BaseNTok}[1]{\textcolor[rgb]{0.25,0.63,0.44}{{#1}}}
    \newcommand{\FloatTok}[1]{\textcolor[rgb]{0.25,0.63,0.44}{{#1}}}
    \newcommand{\CharTok}[1]{\textcolor[rgb]{0.25,0.44,0.63}{{#1}}}
    \newcommand{\StringTok}[1]{\textcolor[rgb]{0.25,0.44,0.63}{{#1}}}
    \newcommand{\CommentTok}[1]{\textcolor[rgb]{0.38,0.63,0.69}{\textit{{#1}}}}
    \newcommand{\OtherTok}[1]{\textcolor[rgb]{0.00,0.44,0.13}{{#1}}}
    \newcommand{\AlertTok}[1]{\textcolor[rgb]{1.00,0.00,0.00}{\textbf{{#1}}}}
    \newcommand{\FunctionTok}[1]{\textcolor[rgb]{0.02,0.16,0.49}{{#1}}}
    \newcommand{\RegionMarkerTok}[1]{{#1}}
    \newcommand{\ErrorTok}[1]{\textcolor[rgb]{1.00,0.00,0.00}{\textbf{{#1}}}}
    \newcommand{\NormalTok}[1]{{#1}}

    % Additional commands for more recent versions of Pandoc
    \newcommand{\ConstantTok}[1]{\textcolor[rgb]{0.53,0.00,0.00}{{#1}}}
    \newcommand{\SpecialCharTok}[1]{\textcolor[rgb]{0.25,0.44,0.63}{{#1}}}
    \newcommand{\VerbatimStringTok}[1]{\textcolor[rgb]{0.25,0.44,0.63}{{#1}}}
    \newcommand{\SpecialStringTok}[1]{\textcolor[rgb]{0.73,0.40,0.53}{{#1}}}
    \newcommand{\ImportTok}[1]{{#1}}
    \newcommand{\DocumentationTok}[1]{\textcolor[rgb]{0.73,0.13,0.13}{\textit{{#1}}}}
    \newcommand{\AnnotationTok}[1]{\textcolor[rgb]{0.38,0.63,0.69}{\textbf{\textit{{#1}}}}}
    \newcommand{\CommentVarTok}[1]{\textcolor[rgb]{0.38,0.63,0.69}{\textbf{\textit{{#1}}}}}
    \newcommand{\VariableTok}[1]{\textcolor[rgb]{0.10,0.09,0.49}{{#1}}}
    \newcommand{\ControlFlowTok}[1]{\textcolor[rgb]{0.00,0.44,0.13}{\textbf{{#1}}}}
    \newcommand{\OperatorTok}[1]{\textcolor[rgb]{0.40,0.40,0.40}{{#1}}}
    \newcommand{\BuiltInTok}[1]{{#1}}
    \newcommand{\ExtensionTok}[1]{{#1}}
    \newcommand{\PreprocessorTok}[1]{\textcolor[rgb]{0.74,0.48,0.00}{{#1}}}
    \newcommand{\AttributeTok}[1]{\textcolor[rgb]{0.49,0.56,0.16}{{#1}}}
    \newcommand{\InformationTok}[1]{\textcolor[rgb]{0.38,0.63,0.69}{\textbf{\textit{{#1}}}}}
    \newcommand{\WarningTok}[1]{\textcolor[rgb]{0.38,0.63,0.69}{\textbf{\textit{{#1}}}}}


    % Define a nice break command that doesn't care if a line doesn't already
    % exist.
    \def\br{\hspace*{\fill} \\* }
    % Math Jax compatibility definitions
    \def\gt{>}
    \def\lt{<}
    \let\Oldtex\TeX
    \let\Oldlatex\LaTeX
    \renewcommand{\TeX}{\textrm{\Oldtex}}
    \renewcommand{\LaTeX}{\textrm{\Oldlatex}}
    % Document parameters
    % Document title
    \title{blatt04-python}
    
    
    
    
    
% Pygments definitions
\makeatletter
\def\PY@reset{\let\PY@it=\relax \let\PY@bf=\relax%
    \let\PY@ul=\relax \let\PY@tc=\relax%
    \let\PY@bc=\relax \let\PY@ff=\relax}
\def\PY@tok#1{\csname PY@tok@#1\endcsname}
\def\PY@toks#1+{\ifx\relax#1\empty\else%
    \PY@tok{#1}\expandafter\PY@toks\fi}
\def\PY@do#1{\PY@bc{\PY@tc{\PY@ul{%
    \PY@it{\PY@bf{\PY@ff{#1}}}}}}}
\def\PY#1#2{\PY@reset\PY@toks#1+\relax+\PY@do{#2}}

\@namedef{PY@tok@w}{\def\PY@tc##1{\textcolor[rgb]{0.73,0.73,0.73}{##1}}}
\@namedef{PY@tok@c}{\let\PY@it=\textit\def\PY@tc##1{\textcolor[rgb]{0.24,0.48,0.48}{##1}}}
\@namedef{PY@tok@cp}{\def\PY@tc##1{\textcolor[rgb]{0.61,0.40,0.00}{##1}}}
\@namedef{PY@tok@k}{\let\PY@bf=\textbf\def\PY@tc##1{\textcolor[rgb]{0.00,0.50,0.00}{##1}}}
\@namedef{PY@tok@kp}{\def\PY@tc##1{\textcolor[rgb]{0.00,0.50,0.00}{##1}}}
\@namedef{PY@tok@kt}{\def\PY@tc##1{\textcolor[rgb]{0.69,0.00,0.25}{##1}}}
\@namedef{PY@tok@o}{\def\PY@tc##1{\textcolor[rgb]{0.40,0.40,0.40}{##1}}}
\@namedef{PY@tok@ow}{\let\PY@bf=\textbf\def\PY@tc##1{\textcolor[rgb]{0.67,0.13,1.00}{##1}}}
\@namedef{PY@tok@nb}{\def\PY@tc##1{\textcolor[rgb]{0.00,0.50,0.00}{##1}}}
\@namedef{PY@tok@nf}{\def\PY@tc##1{\textcolor[rgb]{0.00,0.00,1.00}{##1}}}
\@namedef{PY@tok@nc}{\let\PY@bf=\textbf\def\PY@tc##1{\textcolor[rgb]{0.00,0.00,1.00}{##1}}}
\@namedef{PY@tok@nn}{\let\PY@bf=\textbf\def\PY@tc##1{\textcolor[rgb]{0.00,0.00,1.00}{##1}}}
\@namedef{PY@tok@ne}{\let\PY@bf=\textbf\def\PY@tc##1{\textcolor[rgb]{0.80,0.25,0.22}{##1}}}
\@namedef{PY@tok@nv}{\def\PY@tc##1{\textcolor[rgb]{0.10,0.09,0.49}{##1}}}
\@namedef{PY@tok@no}{\def\PY@tc##1{\textcolor[rgb]{0.53,0.00,0.00}{##1}}}
\@namedef{PY@tok@nl}{\def\PY@tc##1{\textcolor[rgb]{0.46,0.46,0.00}{##1}}}
\@namedef{PY@tok@ni}{\let\PY@bf=\textbf\def\PY@tc##1{\textcolor[rgb]{0.44,0.44,0.44}{##1}}}
\@namedef{PY@tok@na}{\def\PY@tc##1{\textcolor[rgb]{0.41,0.47,0.13}{##1}}}
\@namedef{PY@tok@nt}{\let\PY@bf=\textbf\def\PY@tc##1{\textcolor[rgb]{0.00,0.50,0.00}{##1}}}
\@namedef{PY@tok@nd}{\def\PY@tc##1{\textcolor[rgb]{0.67,0.13,1.00}{##1}}}
\@namedef{PY@tok@s}{\def\PY@tc##1{\textcolor[rgb]{0.73,0.13,0.13}{##1}}}
\@namedef{PY@tok@sd}{\let\PY@it=\textit\def\PY@tc##1{\textcolor[rgb]{0.73,0.13,0.13}{##1}}}
\@namedef{PY@tok@si}{\let\PY@bf=\textbf\def\PY@tc##1{\textcolor[rgb]{0.64,0.35,0.47}{##1}}}
\@namedef{PY@tok@se}{\let\PY@bf=\textbf\def\PY@tc##1{\textcolor[rgb]{0.67,0.36,0.12}{##1}}}
\@namedef{PY@tok@sr}{\def\PY@tc##1{\textcolor[rgb]{0.64,0.35,0.47}{##1}}}
\@namedef{PY@tok@ss}{\def\PY@tc##1{\textcolor[rgb]{0.10,0.09,0.49}{##1}}}
\@namedef{PY@tok@sx}{\def\PY@tc##1{\textcolor[rgb]{0.00,0.50,0.00}{##1}}}
\@namedef{PY@tok@m}{\def\PY@tc##1{\textcolor[rgb]{0.40,0.40,0.40}{##1}}}
\@namedef{PY@tok@gh}{\let\PY@bf=\textbf\def\PY@tc##1{\textcolor[rgb]{0.00,0.00,0.50}{##1}}}
\@namedef{PY@tok@gu}{\let\PY@bf=\textbf\def\PY@tc##1{\textcolor[rgb]{0.50,0.00,0.50}{##1}}}
\@namedef{PY@tok@gd}{\def\PY@tc##1{\textcolor[rgb]{0.63,0.00,0.00}{##1}}}
\@namedef{PY@tok@gi}{\def\PY@tc##1{\textcolor[rgb]{0.00,0.52,0.00}{##1}}}
\@namedef{PY@tok@gr}{\def\PY@tc##1{\textcolor[rgb]{0.89,0.00,0.00}{##1}}}
\@namedef{PY@tok@ge}{\let\PY@it=\textit}
\@namedef{PY@tok@gs}{\let\PY@bf=\textbf}
\@namedef{PY@tok@gp}{\let\PY@bf=\textbf\def\PY@tc##1{\textcolor[rgb]{0.00,0.00,0.50}{##1}}}
\@namedef{PY@tok@go}{\def\PY@tc##1{\textcolor[rgb]{0.44,0.44,0.44}{##1}}}
\@namedef{PY@tok@gt}{\def\PY@tc##1{\textcolor[rgb]{0.00,0.27,0.87}{##1}}}
\@namedef{PY@tok@err}{\def\PY@bc##1{{\setlength{\fboxsep}{\string -\fboxrule}\fcolorbox[rgb]{1.00,0.00,0.00}{1,1,1}{\strut ##1}}}}
\@namedef{PY@tok@kc}{\let\PY@bf=\textbf\def\PY@tc##1{\textcolor[rgb]{0.00,0.50,0.00}{##1}}}
\@namedef{PY@tok@kd}{\let\PY@bf=\textbf\def\PY@tc##1{\textcolor[rgb]{0.00,0.50,0.00}{##1}}}
\@namedef{PY@tok@kn}{\let\PY@bf=\textbf\def\PY@tc##1{\textcolor[rgb]{0.00,0.50,0.00}{##1}}}
\@namedef{PY@tok@kr}{\let\PY@bf=\textbf\def\PY@tc##1{\textcolor[rgb]{0.00,0.50,0.00}{##1}}}
\@namedef{PY@tok@bp}{\def\PY@tc##1{\textcolor[rgb]{0.00,0.50,0.00}{##1}}}
\@namedef{PY@tok@fm}{\def\PY@tc##1{\textcolor[rgb]{0.00,0.00,1.00}{##1}}}
\@namedef{PY@tok@vc}{\def\PY@tc##1{\textcolor[rgb]{0.10,0.09,0.49}{##1}}}
\@namedef{PY@tok@vg}{\def\PY@tc##1{\textcolor[rgb]{0.10,0.09,0.49}{##1}}}
\@namedef{PY@tok@vi}{\def\PY@tc##1{\textcolor[rgb]{0.10,0.09,0.49}{##1}}}
\@namedef{PY@tok@vm}{\def\PY@tc##1{\textcolor[rgb]{0.10,0.09,0.49}{##1}}}
\@namedef{PY@tok@sa}{\def\PY@tc##1{\textcolor[rgb]{0.73,0.13,0.13}{##1}}}
\@namedef{PY@tok@sb}{\def\PY@tc##1{\textcolor[rgb]{0.73,0.13,0.13}{##1}}}
\@namedef{PY@tok@sc}{\def\PY@tc##1{\textcolor[rgb]{0.73,0.13,0.13}{##1}}}
\@namedef{PY@tok@dl}{\def\PY@tc##1{\textcolor[rgb]{0.73,0.13,0.13}{##1}}}
\@namedef{PY@tok@s2}{\def\PY@tc##1{\textcolor[rgb]{0.73,0.13,0.13}{##1}}}
\@namedef{PY@tok@sh}{\def\PY@tc##1{\textcolor[rgb]{0.73,0.13,0.13}{##1}}}
\@namedef{PY@tok@s1}{\def\PY@tc##1{\textcolor[rgb]{0.73,0.13,0.13}{##1}}}
\@namedef{PY@tok@mb}{\def\PY@tc##1{\textcolor[rgb]{0.40,0.40,0.40}{##1}}}
\@namedef{PY@tok@mf}{\def\PY@tc##1{\textcolor[rgb]{0.40,0.40,0.40}{##1}}}
\@namedef{PY@tok@mh}{\def\PY@tc##1{\textcolor[rgb]{0.40,0.40,0.40}{##1}}}
\@namedef{PY@tok@mi}{\def\PY@tc##1{\textcolor[rgb]{0.40,0.40,0.40}{##1}}}
\@namedef{PY@tok@il}{\def\PY@tc##1{\textcolor[rgb]{0.40,0.40,0.40}{##1}}}
\@namedef{PY@tok@mo}{\def\PY@tc##1{\textcolor[rgb]{0.40,0.40,0.40}{##1}}}
\@namedef{PY@tok@ch}{\let\PY@it=\textit\def\PY@tc##1{\textcolor[rgb]{0.24,0.48,0.48}{##1}}}
\@namedef{PY@tok@cm}{\let\PY@it=\textit\def\PY@tc##1{\textcolor[rgb]{0.24,0.48,0.48}{##1}}}
\@namedef{PY@tok@cpf}{\let\PY@it=\textit\def\PY@tc##1{\textcolor[rgb]{0.24,0.48,0.48}{##1}}}
\@namedef{PY@tok@c1}{\let\PY@it=\textit\def\PY@tc##1{\textcolor[rgb]{0.24,0.48,0.48}{##1}}}
\@namedef{PY@tok@cs}{\let\PY@it=\textit\def\PY@tc##1{\textcolor[rgb]{0.24,0.48,0.48}{##1}}}

\def\PYZbs{\char`\\}
\def\PYZus{\char`\_}
\def\PYZob{\char`\{}
\def\PYZcb{\char`\}}
\def\PYZca{\char`\^}
\def\PYZam{\char`\&}
\def\PYZlt{\char`\<}
\def\PYZgt{\char`\>}
\def\PYZsh{\char`\#}
\def\PYZpc{\char`\%}
\def\PYZdl{\char`\$}
\def\PYZhy{\char`\-}
\def\PYZsq{\char`\'}
\def\PYZdq{\char`\"}
\def\PYZti{\char`\~}
% for compatibility with earlier versions
\def\PYZat{@}
\def\PYZlb{[}
\def\PYZrb{]}
\makeatother


    % For linebreaks inside Verbatim environment from package fancyvrb.
    \makeatletter
        \newbox\Wrappedcontinuationbox
        \newbox\Wrappedvisiblespacebox
        \newcommand*\Wrappedvisiblespace {\textcolor{red}{\textvisiblespace}}
        \newcommand*\Wrappedcontinuationsymbol {\textcolor{red}{\llap{\tiny$\m@th\hookrightarrow$}}}
        \newcommand*\Wrappedcontinuationindent {3ex }
        \newcommand*\Wrappedafterbreak {\kern\Wrappedcontinuationindent\copy\Wrappedcontinuationbox}
        % Take advantage of the already applied Pygments mark-up to insert
        % potential linebreaks for TeX processing.
        %        {, <, #, %, $, ' and ": go to next line.
        %        _, }, ^, &, >, - and ~: stay at end of broken line.
        % Use of \textquotesingle for straight quote.
        \newcommand*\Wrappedbreaksatspecials {%
            \def\PYGZus{\discretionary{\char`\_}{\Wrappedafterbreak}{\char`\_}}%
            \def\PYGZob{\discretionary{}{\Wrappedafterbreak\char`\{}{\char`\{}}%
            \def\PYGZcb{\discretionary{\char`\}}{\Wrappedafterbreak}{\char`\}}}%
            \def\PYGZca{\discretionary{\char`\^}{\Wrappedafterbreak}{\char`\^}}%
            \def\PYGZam{\discretionary{\char`\&}{\Wrappedafterbreak}{\char`\&}}%
            \def\PYGZlt{\discretionary{}{\Wrappedafterbreak\char`\<}{\char`\<}}%
            \def\PYGZgt{\discretionary{\char`\>}{\Wrappedafterbreak}{\char`\>}}%
            \def\PYGZsh{\discretionary{}{\Wrappedafterbreak\char`\#}{\char`\#}}%
            \def\PYGZpc{\discretionary{}{\Wrappedafterbreak\char`\%}{\char`\%}}%
            \def\PYGZdl{\discretionary{}{\Wrappedafterbreak\char`\$}{\char`\$}}%
            \def\PYGZhy{\discretionary{\char`\-}{\Wrappedafterbreak}{\char`\-}}%
            \def\PYGZsq{\discretionary{}{\Wrappedafterbreak\textquotesingle}{\textquotesingle}}%
            \def\PYGZdq{\discretionary{}{\Wrappedafterbreak\char`\"}{\char`\"}}%
            \def\PYGZti{\discretionary{\char`\~}{\Wrappedafterbreak}{\char`\~}}%
        }
        % Some characters . , ; ? ! / are not pygmentized.
        % This macro makes them "active" and they will insert potential linebreaks
        \newcommand*\Wrappedbreaksatpunct {%
            \lccode`\~`\.\lowercase{\def~}{\discretionary{\hbox{\char`\.}}{\Wrappedafterbreak}{\hbox{\char`\.}}}%
            \lccode`\~`\,\lowercase{\def~}{\discretionary{\hbox{\char`\,}}{\Wrappedafterbreak}{\hbox{\char`\,}}}%
            \lccode`\~`\;\lowercase{\def~}{\discretionary{\hbox{\char`\;}}{\Wrappedafterbreak}{\hbox{\char`\;}}}%
            \lccode`\~`\:\lowercase{\def~}{\discretionary{\hbox{\char`\:}}{\Wrappedafterbreak}{\hbox{\char`\:}}}%
            \lccode`\~`\?\lowercase{\def~}{\discretionary{\hbox{\char`\?}}{\Wrappedafterbreak}{\hbox{\char`\?}}}%
            \lccode`\~`\!\lowercase{\def~}{\discretionary{\hbox{\char`\!}}{\Wrappedafterbreak}{\hbox{\char`\!}}}%
            \lccode`\~`\/\lowercase{\def~}{\discretionary{\hbox{\char`\/}}{\Wrappedafterbreak}{\hbox{\char`\/}}}%
            \catcode`\.\active
            \catcode`\,\active
            \catcode`\;\active
            \catcode`\:\active
            \catcode`\?\active
            \catcode`\!\active
            \catcode`\/\active
            \lccode`\~`\~
        }
    \makeatother

    \let\OriginalVerbatim=\Verbatim
    \makeatletter
    \renewcommand{\Verbatim}[1][1]{%
        %\parskip\z@skip
        \sbox\Wrappedcontinuationbox {\Wrappedcontinuationsymbol}%
        \sbox\Wrappedvisiblespacebox {\FV@SetupFont\Wrappedvisiblespace}%
        \def\FancyVerbFormatLine ##1{\hsize\linewidth
            \vtop{\raggedright\hyphenpenalty\z@\exhyphenpenalty\z@
                \doublehyphendemerits\z@\finalhyphendemerits\z@
                \strut ##1\strut}%
        }%
        % If the linebreak is at a space, the latter will be displayed as visible
        % space at end of first line, and a continuation symbol starts next line.
        % Stretch/shrink are however usually zero for typewriter font.
        \def\FV@Space {%
            \nobreak\hskip\z@ plus\fontdimen3\font minus\fontdimen4\font
            \discretionary{\copy\Wrappedvisiblespacebox}{\Wrappedafterbreak}
            {\kern\fontdimen2\font}%
        }%

        % Allow breaks at special characters using \PYG... macros.
        \Wrappedbreaksatspecials
        % Breaks at punctuation characters . , ; ? ! and / need catcode=\active
        \OriginalVerbatim[#1,codes*=\Wrappedbreaksatpunct]%
    }
    \makeatother

    % Exact colors from NB
    \definecolor{incolor}{HTML}{303F9F}
    \definecolor{outcolor}{HTML}{D84315}
    \definecolor{cellborder}{HTML}{CFCFCF}
    \definecolor{cellbackground}{HTML}{F7F7F7}

    % prompt
    \makeatletter
    \newcommand{\boxspacing}{\kern\kvtcb@left@rule\kern\kvtcb@boxsep}
    \makeatother
    \newcommand{\prompt}[4]{
        {\ttfamily\llap{{\color{#2}[#3]:\hspace{3pt}#4}}\vspace{-\baselineskip}}
    }
    

    
    % Prevent overflowing lines due to hard-to-break entities
    \sloppy
    % Setup hyperref package
    \hypersetup{
      breaklinks=true,  % so long urls are correctly broken across lines
      colorlinks=true,
      urlcolor=urlcolor,
      linkcolor=linkcolor,
      citecolor=citecolor,
      }
    % Slightly bigger margins than the latex defaults
    
    \geometry{verbose,tmargin=1in,bmargin=1in,lmargin=1in,rmargin=1in}
    
    

\begin{document}
    
    \maketitle
    
    

    
    \hypertarget{datenstrukturen-und-algorithmen}{%
\section{Datenstrukturen und
Algorithmen}\label{datenstrukturen-und-algorithmen}}

\hypertarget{praktische-aufgabe-2}{%
\subsection{Praktische Aufgabe 2}\label{praktische-aufgabe-2}}

In dieser praktischen Aufgabe werden Sie sich mit Dynamischem
Programmieren, sowie einfachen Sortierverfahren beschäftigen. Diese
Aufgabe dient dazu die Konzepte aus der Vorlesung zu festigen und soll
Ihnen dabei helfen ein Gefühl für die Funktionsweise der
Sortieralgorithmen zu entwickeln.

Die Abgaben werden mit der \texttt{nbgrader} Erweiterung korrigiert. Das
System erwartet, dass der Code zum Lösen der Aufgaben nach der
\texttt{\#YOUR\ CODE\ HERE} Anweisung kommt. Außerdem darf die
Zellenreihenfolge nicht geändert werden. Damit Sie selbst Ihre
Lösungsvorschläge validieren können, werden Ihnen Unittests zur
Verfügung gestellt. Beachten Sie, dass diese Tests keine Garantie sind
für das Erreichen der vollen Punktzahl, da Sie nur einen Teil der
Funktionalität überprüfen.

Wichtig: Füllen Sie auch die erste Zelle mit dem Titel Abgabeteam
vollständig aus. Dies ermöglicht uns auch bei technischen Problemen die
Abgaben eindeutig zuordnen zu können. Ändern Sie außerdem nicht den
Namen der Datei.

\textbf{Übersicht der Aufgaben} (20 Punkte):

\begin{enumerate}
\def\labelenumi{\arabic{enumi}.}
\tightlist
\item
  \textbf{Dynamisches Programmieren} - insgesamt: 10 Punkte

  \begin{itemize}
  \tightlist
  \item
    Matrix-Multiplikation - 10P.
  \end{itemize}
\item
  \textbf{Einfache Sortierverfahren} - insgesamt: 10 Punkte

  \begin{itemize}
  \tightlist
  \item
    Selection-Sort - 4P.
  \item
    Insertion-Sort - 4P.
  \item
    Vergleich von Operationen - 2P.
  \end{itemize}
\end{enumerate}

    \hypertarget{abgabeteam}{%
\subsection{Abgabeteam}\label{abgabeteam}}

Bitte füllen Sie die untenstehende Zelle aus mit

Nummer des Tutoriums,

Mohammed Al-Laktah 419664,

Salah Atallah 414867,

(Vorname Nachname Matrikelnummer 3)

    Tutorium Musterlösung

Max Mustermann 123456

Erika Mustermann 123457

(Paul Mustermann 123458)

    \hypertarget{module-importieren}{%
\subsection{Module importieren}\label{module-importieren}}

Zuerst werden die benötigten Module importiert. Sie dürfen keine
weiteren Module impotieren.

Wenn in Ihrer Entwickungsumbegung (z.B Deepnote) bestimmte Module nicht
verfügbar sind, dann kommentieren Sie die erste Zeile aus um die Module
temporär in der Umgebung zu installieren.

    \begin{tcolorbox}[breakable, size=fbox, boxrule=1pt, pad at break*=1mm,colback=cellbackground, colframe=cellborder]
\prompt{In}{incolor}{80}{\boxspacing}
\begin{Verbatim}[commandchars=\\\{\}]
\PY{c+c1}{\PYZsh{}\PYZpc{}pip install pandas, rwth\PYZus{}nb, nose}

\PY{c+c1}{\PYZsh{} unittests}
\PY{k+kn}{from} \PY{n+nn}{nose}\PY{n+nn}{.}\PY{n+nn}{tools} \PY{k+kn}{import} \PY{n}{assert\PYZus{}equal}

\PY{c+c1}{\PYZsh{} measuring time}
\PY{k+kn}{from} \PY{n+nn}{time} \PY{k+kn}{import} \PY{n}{time}

\PY{c+c1}{\PYZsh{} random numbers}
\PY{k+kn}{from} \PY{n+nn}{random} \PY{k+kn}{import} \PY{n}{randint}

\PY{c+c1}{\PYZsh{} plotting}
\PY{k+kn}{import} \PY{n+nn}{pandas} \PY{k}{as} \PY{n+nn}{pd}
\PY{k+kn}{import} \PY{n+nn}{rwth\PYZus{}nb}\PY{n+nn}{.}\PY{n+nn}{plots}\PY{n+nn}{.}\PY{n+nn}{mpl\PYZus{}decorations} \PY{k}{as} \PY{n+nn}{rwth\PYZus{}plt}
\end{Verbatim}
\end{tcolorbox}

    \hypertarget{dynamisches-programmieren}{%
\section{Dynamisches Programmieren}\label{dynamisches-programmieren}}

Im Folgenden sollen Sie den in der Vorlesung vorgestellten Code
modifizieren, um nicht nur die minimale Anzahl an Skalarmultiplikationen
auszurechnen, sondern auch die korrekte Klammerung auszugeben.

    \hypertarget{a-matrix_multiplication---10p.}{%
\subsection{a) matrix\_multiplication() -
10P.}\label{a-matrix_multiplication---10p.}}

Erweitern Sie den in der Vorlesung vorgestellten Code um eine Matrix
\texttt{S}, in der Sie zusätzliche speichern, welche Kombination der
Teillösungen Sie in jeder Teillösung ausgewählt haben. Diese Matrix soll
für jede Teillösung die Klammerung der Matritzen mit ihren Namen
speichern, die zu einer minimalen Anzahl an Skalarmultiplikationen
führt.

Bitte beachten Sie zusätzlich, dass Sie jetzt nicht einen Vektor
\texttt{r} als Eingabe bekommen, sondern ein Dictionary \texttt{R} in
dem der Name, sowie die Dimension der Matrix als Key und Value angegeben
werden. Initialisieren Sie zunächst den Vektor \texttt{r}, sowie die
Matritzen \texttt{M} und \texttt{S} mit der richtigen Werten, bevor Sie
mit der Berechnung beginnen. Die Matritzen in \texttt{R} sollen der
Reihe nach miteinander multipliziert werden. Die Matrix \texttt{S} soll
die Namen der Matritzen, sowie deren korrekte Klammerung beinhalten.
Geben Sie am Ende der Berechnung sowohl die Matrix \texttt{M}, als auch
\texttt{S} aus.

Die folgende Abbildung zeigt das Verhalten der Funktion
\emph{matrix\_multiplication()} an einem Beispiel:

\begin{verbatim}
Eingabe:
R = {'A': (10, 1), 'B': (1, 10), 'C': (10, 1)}

Ausgabe:
M = [[0, 100, 20], [None, 0, 10], [None, None, 0]]
S = [['A', '(AB)', '(A(BC))'], [None, 'B', '(BC)'], [None, None, 'C']]


Eingabe:
R = {'A': (30, 1), 'B': (1, 40), 'C': (40, 10), 'D': (10, 25)}

Ausgabe:
M = [[0, 1200, 700, 1400], [None, 0, 400, 650], [None, None, 0, 10000], [None, None, None, 0]]
S = [['A', '(AB)', '(A(BC))', '(A((BC)D))'], [None, 'B', '(BC)', '((BC)D)'], [None, None, 'C', '(CD)'], [None, None, None, 'D']]
\end{verbatim}

Sie dürfen/sollen in dieser Aufgabe die Listen benutzen die von Python
zur Verfügung gestellt werden.

    \begin{tcolorbox}[breakable, size=fbox, boxrule=1pt, pad at break*=1mm,colback=cellbackground, colframe=cellborder]
\prompt{In}{incolor}{81}{\boxspacing}
\begin{Verbatim}[commandchars=\\\{\}]
\PY{k}{def} \PY{n+nf}{matrix\PYZus{}multiplication}\PY{p}{(}\PY{n}{R}\PY{p}{)}\PY{p}{:}
    \PY{n}{n} \PY{o}{=} \PY{n+nb}{len}\PY{p}{(}\PY{n}{R}\PY{p}{)}    
    \PY{n}{M} \PY{o}{=} \PY{p}{[}\PY{n}{x}\PY{p}{[}\PY{p}{:}\PY{p}{]} \PY{k}{for} \PY{n}{x} \PY{o+ow}{in} \PY{n}{n} \PY{o}{*} \PY{p}{[}\PY{n}{n} \PY{o}{*} \PY{p}{[}\PY{k+kc}{None}\PY{p}{]}\PY{p}{]}\PY{p}{]}
    \PY{n}{S} \PY{o}{=} \PY{p}{[}\PY{n}{x}\PY{p}{[}\PY{p}{:}\PY{p}{]} \PY{k}{for} \PY{n}{x} \PY{o+ow}{in} \PY{n}{n} \PY{o}{*} \PY{p}{[}\PY{n}{n} \PY{o}{*} \PY{p}{[}\PY{k+kc}{None}\PY{p}{]}\PY{p}{]}\PY{p}{]}

    \PY{c+c1}{\PYZsh{} YOUR CODE HERE 2.2}
    \PY{n}{r} \PY{o}{=} \PY{p}{[}\PY{n}{v}\PY{p}{[}\PY{l+m+mi}{0}\PY{p}{]} \PY{k}{for} \PY{n}{v} \PY{o+ow}{in} \PY{n}{R}\PY{o}{.}\PY{n}{values}\PY{p}{(}\PY{p}{)}\PY{p}{]} \PY{o}{+} \PY{p}{[}\PY{n+nb}{list}\PY{p}{(}\PY{n}{R}\PY{o}{.}\PY{n}{values}\PY{p}{(}\PY{p}{)}\PY{p}{)}\PY{p}{[}\PY{o}{\PYZhy{}}\PY{l+m+mi}{1}\PY{p}{]}\PY{p}{[}\PY{l+m+mi}{1}\PY{p}{]}\PY{p}{]}
    
    \PY{k}{for} \PY{n}{i} \PY{o+ow}{in} \PY{n+nb}{range}\PY{p}{(}\PY{l+m+mi}{0}\PY{p}{,} \PY{n}{n}\PY{p}{)}\PY{p}{:}
        \PY{n}{M}\PY{p}{[}\PY{n}{i}\PY{p}{]}\PY{p}{[}\PY{n}{i}\PY{p}{]} \PY{o}{=} \PY{l+m+mi}{0}
        \PY{n}{S}\PY{p}{[}\PY{n}{i}\PY{p}{]}\PY{p}{[}\PY{n}{i}\PY{p}{]} \PY{o}{=} \PY{n+nb}{list}\PY{p}{(}\PY{n}{R}\PY{o}{.}\PY{n}{keys}\PY{p}{(}\PY{p}{)}\PY{p}{)}\PY{p}{[}\PY{n}{i}\PY{p}{]}
    
    \PY{k}{def} \PY{n+nf}{stringify}\PY{p}{(}\PY{n}{a}\PY{p}{,}\PY{n}{b}\PY{p}{)}\PY{p}{:}
        \PY{k}{return} \PY{l+s+sa}{f}\PY{l+s+s1}{\PYZsq{}}\PY{l+s+s1}{(}\PY{l+s+si}{\PYZob{}}\PY{n}{a}\PY{l+s+si}{\PYZcb{}}\PY{l+s+si}{\PYZob{}}\PY{n}{b}\PY{l+s+si}{\PYZcb{}}\PY{l+s+s1}{)}\PY{l+s+s1}{\PYZsq{}}
    
    \PY{k}{for} \PY{n}{k} \PY{o+ow}{in} \PY{n+nb}{range}\PY{p}{(}\PY{l+m+mi}{1}\PY{p}{,} \PY{n}{n}\PY{p}{)}\PY{p}{:}
        \PY{k}{for} \PY{n}{i} \PY{o+ow}{in} \PY{n+nb}{range}\PY{p}{(}\PY{l+m+mi}{0}\PY{p}{,} \PY{n}{n}\PY{o}{\PYZhy{}}\PY{n}{k}\PY{p}{)}\PY{p}{:}
            \PY{n}{values} \PY{o}{=} \PY{p}{[}\PY{p}{]}
            \PY{k}{for} \PY{n}{j} \PY{o+ow}{in} \PY{n+nb}{range}\PY{p}{(}\PY{n}{i}\PY{p}{,} \PY{n}{i} \PY{o}{+} \PY{n}{k}\PY{p}{)}\PY{p}{:}
                \PY{n}{values}\PY{o}{.}\PY{n}{append}\PY{p}{(}
                    \PY{p}{(}
                        \PY{n}{M}\PY{p}{[}\PY{n}{i}\PY{p}{]}\PY{p}{[}\PY{n}{j}\PY{p}{]} \PY{o}{+} \PY{n}{M}\PY{p}{[}\PY{n}{j}\PY{o}{+}\PY{l+m+mi}{1}\PY{p}{]}\PY{p}{[}\PY{n}{i}\PY{o}{+}\PY{n}{k}\PY{p}{]} \PY{o}{+} \PY{n}{r}\PY{p}{[}\PY{n}{i}\PY{p}{]} \PY{o}{*} \PY{n}{r}\PY{p}{[}\PY{n}{j}\PY{o}{+}\PY{l+m+mi}{1}\PY{p}{]} \PY{o}{*} \PY{n}{r}\PY{p}{[}\PY{n}{i}\PY{o}{+}\PY{n}{k}\PY{o}{+}\PY{l+m+mi}{1}\PY{p}{]}\PY{p}{,}
                        \PY{n}{stringify}\PY{p}{(}\PY{n}{S}\PY{p}{[}\PY{n}{i}\PY{p}{]}\PY{p}{[}\PY{n}{j}\PY{p}{]}\PY{p}{,} \PY{n}{S}\PY{p}{[}\PY{n}{j}\PY{o}{+}\PY{l+m+mi}{1}\PY{p}{]}\PY{p}{[}\PY{n}{i}\PY{o}{+}\PY{n}{k}\PY{p}{]}\PY{p}{)}
                    \PY{p}{)}
                \PY{p}{)}
            \PY{n}{smallest} \PY{o}{=} \PY{l+m+mi}{0}
            \PY{k}{for} \PY{n}{l}\PY{p}{,} \PY{p}{(}\PY{n}{value}\PY{p}{,} \PY{n}{\PYZus{}}\PY{p}{)} \PY{o+ow}{in} \PY{n+nb}{enumerate}\PY{p}{(}\PY{n}{values}\PY{p}{)}\PY{p}{:}
                \PY{k}{if} \PY{n}{value} \PY{o}{\PYZlt{}} \PY{n}{values}\PY{p}{[}\PY{n}{smallest}\PY{p}{]}\PY{p}{[}\PY{l+m+mi}{0}\PY{p}{]}\PY{p}{:}
                    \PY{n}{smallest} \PY{o}{=} \PY{n}{l}
            \PY{n}{M}\PY{p}{[}\PY{n}{i}\PY{p}{]}\PY{p}{[}\PY{n}{i}\PY{o}{+}\PY{n}{k}\PY{p}{]} \PY{o}{=} \PY{n}{values}\PY{p}{[}\PY{n}{smallest}\PY{p}{]}\PY{p}{[}\PY{l+m+mi}{0}\PY{p}{]}
            \PY{n}{S}\PY{p}{[}\PY{n}{i}\PY{p}{]}\PY{p}{[}\PY{n}{i}\PY{o}{+}\PY{n}{k}\PY{p}{]} \PY{o}{=} \PY{n}{values}\PY{p}{[}\PY{n}{smallest}\PY{p}{]}\PY{p}{[}\PY{l+m+mi}{1}\PY{p}{]}
    \PY{c+c1}{\PYZsh{} raise NotImplementedError()}

    \PY{k}{return} \PY{n}{M}\PY{p}{,} \PY{n}{S}
\end{Verbatim}
\end{tcolorbox}

    \begin{tcolorbox}[breakable, size=fbox, boxrule=1pt, pad at break*=1mm,colback=cellbackground, colframe=cellborder]
\prompt{In}{incolor}{82}{\boxspacing}
\begin{Verbatim}[commandchars=\\\{\}]
\PY{c+c1}{\PYZsh{}general case}
\PY{c+c1}{\PYZsh{}test1}
\PY{n}{R} \PY{o}{=} \PY{p}{\PYZob{}}
    \PY{l+s+s2}{\PYZdq{}}\PY{l+s+s2}{A}\PY{l+s+s2}{\PYZdq{}}\PY{p}{:} \PY{p}{(}\PY{l+m+mi}{10}\PY{p}{,} \PY{l+m+mi}{1}\PY{p}{)}\PY{p}{,}
    \PY{l+s+s2}{\PYZdq{}}\PY{l+s+s2}{B}\PY{l+s+s2}{\PYZdq{}}\PY{p}{:} \PY{p}{(}\PY{l+m+mi}{1}\PY{p}{,} \PY{l+m+mi}{10}\PY{p}{)}\PY{p}{,}
    \PY{l+s+s2}{\PYZdq{}}\PY{l+s+s2}{C}\PY{l+s+s2}{\PYZdq{}}\PY{p}{:} \PY{p}{(}\PY{l+m+mi}{10}\PY{p}{,} \PY{l+m+mi}{1}\PY{p}{)}\PY{p}{,}
\PY{p}{\PYZcb{}}
\PY{n}{M}\PY{p}{,} \PY{n}{S} \PY{o}{=} \PY{n}{matrix\PYZus{}multiplication}\PY{p}{(}\PY{n}{R}\PY{p}{)}
\PY{n}{M\PYZus{}groundtruth} \PY{o}{=} \PY{p}{[}\PY{p}{[}\PY{l+m+mi}{0}\PY{p}{,} \PY{l+m+mi}{100}\PY{p}{,} \PY{l+m+mi}{20}\PY{p}{]}\PY{p}{,} \PY{p}{[}\PY{k+kc}{None}\PY{p}{,} \PY{l+m+mi}{0}\PY{p}{,} \PY{l+m+mi}{10}\PY{p}{]}\PY{p}{,} \PY{p}{[}\PY{k+kc}{None}\PY{p}{,} \PY{k+kc}{None}\PY{p}{,} \PY{l+m+mi}{0}\PY{p}{]}\PY{p}{]}
\PY{n}{S\PYZus{}groundtruth} \PY{o}{=} \PY{p}{[}\PY{p}{[}\PY{l+s+s1}{\PYZsq{}}\PY{l+s+s1}{A}\PY{l+s+s1}{\PYZsq{}}\PY{p}{,} \PY{l+s+s1}{\PYZsq{}}\PY{l+s+s1}{(AB)}\PY{l+s+s1}{\PYZsq{}}\PY{p}{,} \PY{l+s+s1}{\PYZsq{}}\PY{l+s+s1}{(A(BC))}\PY{l+s+s1}{\PYZsq{}}\PY{p}{]}\PY{p}{,} \PY{p}{[}\PY{k+kc}{None}\PY{p}{,} \PY{l+s+s1}{\PYZsq{}}\PY{l+s+s1}{B}\PY{l+s+s1}{\PYZsq{}}\PY{p}{,} \PY{l+s+s1}{\PYZsq{}}\PY{l+s+s1}{(BC)}\PY{l+s+s1}{\PYZsq{}}\PY{p}{]}\PY{p}{,} \PY{p}{[}\PY{k+kc}{None}\PY{p}{,} \PY{k+kc}{None}\PY{p}{,} \PY{l+s+s1}{\PYZsq{}}\PY{l+s+s1}{C}\PY{l+s+s1}{\PYZsq{}}\PY{p}{]}\PY{p}{]}
\PY{n}{assert\PYZus{}equal}\PY{p}{(}\PY{n+nb}{str}\PY{p}{(}\PY{n}{M}\PY{p}{)}\PY{p}{,} \PY{n+nb}{str}\PY{p}{(}\PY{n}{M\PYZus{}groundtruth}\PY{p}{)}\PY{p}{)}
\PY{n}{assert\PYZus{}equal}\PY{p}{(}\PY{n+nb}{str}\PY{p}{(}\PY{n}{S}\PY{p}{)}\PY{p}{,} \PY{n+nb}{str}\PY{p}{(}\PY{n}{S\PYZus{}groundtruth}\PY{p}{)}\PY{p}{)}

\PY{c+c1}{\PYZsh{}test2}
\PY{n}{R} \PY{o}{=} \PY{p}{\PYZob{}}
    \PY{l+s+s2}{\PYZdq{}}\PY{l+s+s2}{A}\PY{l+s+s2}{\PYZdq{}}\PY{p}{:} \PY{p}{(}\PY{l+m+mi}{20}\PY{p}{,} \PY{l+m+mi}{5}\PY{p}{)}\PY{p}{,}
    \PY{l+s+s2}{\PYZdq{}}\PY{l+s+s2}{B}\PY{l+s+s2}{\PYZdq{}}\PY{p}{:} \PY{p}{(}\PY{l+m+mi}{5}\PY{p}{,} \PY{l+m+mi}{70}\PY{p}{)}\PY{p}{,}
    \PY{l+s+s2}{\PYZdq{}}\PY{l+s+s2}{C}\PY{l+s+s2}{\PYZdq{}}\PY{p}{:} \PY{p}{(}\PY{l+m+mi}{70}\PY{p}{,} \PY{l+m+mi}{5}\PY{p}{)}\PY{p}{,}
    \PY{l+s+s2}{\PYZdq{}}\PY{l+s+s2}{D}\PY{l+s+s2}{\PYZdq{}}\PY{p}{:} \PY{p}{(}\PY{l+m+mi}{5}\PY{p}{,} \PY{l+m+mi}{5}\PY{p}{)}\PY{p}{,}
\PY{p}{\PYZcb{}}
\PY{n}{M}\PY{p}{,} \PY{n}{S} \PY{o}{=} \PY{n}{matrix\PYZus{}multiplication}\PY{p}{(}\PY{n}{R}\PY{p}{)}
\PY{n}{M\PYZus{}groundtruth} \PY{o}{=} \PY{p}{[}\PY{p}{[}\PY{l+m+mi}{0}\PY{p}{,} \PY{l+m+mi}{7000}\PY{p}{,} \PY{l+m+mi}{2250}\PY{p}{,} \PY{l+m+mi}{2375}\PY{p}{]}\PY{p}{,} \PY{p}{[}\PY{k+kc}{None}\PY{p}{,} \PY{l+m+mi}{0}\PY{p}{,} \PY{l+m+mi}{1750}\PY{p}{,} \PY{l+m+mi}{1875}\PY{p}{]}\PY{p}{,} \PY{p}{[}\PY{k+kc}{None}\PY{p}{,} \PY{k+kc}{None}\PY{p}{,} \PY{l+m+mi}{0}\PY{p}{,} \PY{l+m+mi}{1750}\PY{p}{]}\PY{p}{,} \PY{p}{[}\PY{k+kc}{None}\PY{p}{,} \PY{k+kc}{None}\PY{p}{,} \PY{k+kc}{None}\PY{p}{,} \PY{l+m+mi}{0}\PY{p}{]}\PY{p}{]}
\PY{n}{S\PYZus{}groundtruth} \PY{o}{=} \PY{p}{[}\PY{p}{[}\PY{l+s+s1}{\PYZsq{}}\PY{l+s+s1}{A}\PY{l+s+s1}{\PYZsq{}}\PY{p}{,} \PY{l+s+s1}{\PYZsq{}}\PY{l+s+s1}{(AB)}\PY{l+s+s1}{\PYZsq{}}\PY{p}{,} \PY{l+s+s1}{\PYZsq{}}\PY{l+s+s1}{(A(BC))}\PY{l+s+s1}{\PYZsq{}}\PY{p}{,} \PY{l+s+s1}{\PYZsq{}}\PY{l+s+s1}{(A((BC)D))}\PY{l+s+s1}{\PYZsq{}}\PY{p}{]}\PY{p}{,} \PY{p}{[}\PY{k+kc}{None}\PY{p}{,} \PY{l+s+s1}{\PYZsq{}}\PY{l+s+s1}{B}\PY{l+s+s1}{\PYZsq{}}\PY{p}{,} \PY{l+s+s1}{\PYZsq{}}\PY{l+s+s1}{(BC)}\PY{l+s+s1}{\PYZsq{}}\PY{p}{,} \PY{l+s+s1}{\PYZsq{}}\PY{l+s+s1}{((BC)D)}\PY{l+s+s1}{\PYZsq{}}\PY{p}{]}\PY{p}{,} \PY{p}{[}\PY{k+kc}{None}\PY{p}{,} \PY{k+kc}{None}\PY{p}{,} \PY{l+s+s1}{\PYZsq{}}\PY{l+s+s1}{C}\PY{l+s+s1}{\PYZsq{}}\PY{p}{,} \PY{l+s+s1}{\PYZsq{}}\PY{l+s+s1}{(CD)}\PY{l+s+s1}{\PYZsq{}}\PY{p}{]}\PY{p}{,} \PY{p}{[}\PY{k+kc}{None}\PY{p}{,} \PY{k+kc}{None}\PY{p}{,} \PY{k+kc}{None}\PY{p}{,} \PY{l+s+s1}{\PYZsq{}}\PY{l+s+s1}{D}\PY{l+s+s1}{\PYZsq{}}\PY{p}{]}\PY{p}{]}
\PY{n}{assert\PYZus{}equal}\PY{p}{(}\PY{n+nb}{str}\PY{p}{(}\PY{n}{M}\PY{p}{)}\PY{p}{,} \PY{n+nb}{str}\PY{p}{(}\PY{n}{M\PYZus{}groundtruth}\PY{p}{)}\PY{p}{)}
\PY{n}{assert\PYZus{}equal}\PY{p}{(}\PY{n+nb}{str}\PY{p}{(}\PY{n}{S}\PY{p}{)}\PY{p}{,} \PY{n+nb}{str}\PY{p}{(}\PY{n}{S\PYZus{}groundtruth}\PY{p}{)}\PY{p}{)}
\end{Verbatim}
\end{tcolorbox}

    \hypertarget{einfache-sortierverfahren}{%
\section{Einfache Sortierverfahren}\label{einfache-sortierverfahren}}

Der folgende Code implementiert das Bubble-Sort Verfahren in der Methode
\emph{\texttt{bubble\_sort()}}. Beachten Sie, dass als Datenstruktur die
vorimplementierten Listen von Python verwendet werden. Zusätzlich werden
die Zahlen nicht aufsteigend, sondern absteigend sortiert.

Ihre Aufgabe ist es zwei weitere Sortierverfahren, Insertion-Sort und
Selection-Sort, zu implementieren. Dies soll in den jeweiligen Methode
\emph{\texttt{selection\_sort()}} und \emph{\texttt{insertion\_sort()}}
passieren. Anschließend sollen die Algorithmen bezüglich ihrer Laufzeit,
sowie der Anzahl an Vergleichs- und Kopieroperationen verglichen werden.

Beachten Sie, dass die Methoden die übergebenen Listen direkt
modifizieren sollen, weshalb die sortierte Liste nicht zurückgegeben
werden muss.

Die zwei Variablen \texttt{cmp\_ops} und \texttt{cpy\_ops} werden
benutzt um die Anzahl von Vergleichs- und Kopieroperationen innerhalb
eines Aufrufs zu zählen. Vergleiche von Indizes werden in der folgenden
Analyse vernachlässigt. Die Werte für die Operationen werden nach dem
Sortieren zurückgegeben und können für die Auswertung im letzten
Aufgabenteil genutzt werden.

    \begin{tcolorbox}[breakable, size=fbox, boxrule=1pt, pad at break*=1mm,colback=cellbackground, colframe=cellborder]
\prompt{In}{incolor}{83}{\boxspacing}
\begin{Verbatim}[commandchars=\\\{\}]
\PY{k}{def} \PY{n+nf}{bubble\PYZus{}sort}\PY{p}{(}\PY{n}{a}\PY{p}{)}\PY{p}{:}
    \PY{n}{cmp\PYZus{}ops} \PY{o}{=} \PY{l+m+mi}{0}  \PY{c+c1}{\PYZsh{} number of compare operations}
    \PY{n}{cpy\PYZus{}ops} \PY{o}{=} \PY{l+m+mi}{0}  \PY{c+c1}{\PYZsh{} number of copy operations}

    \PY{n}{i} \PY{o}{=} \PY{l+m+mi}{0}
    \PY{n}{n} \PY{o}{=} \PY{n+nb}{len}\PY{p}{(}\PY{n}{a}\PY{p}{)}
    \PY{k}{while} \PY{n}{i} \PY{o}{\PYZlt{}} \PY{n}{n} \PY{o}{\PYZhy{}} \PY{l+m+mi}{1}\PY{p}{:}
        \PY{n}{j} \PY{o}{=} \PY{n}{n} \PY{o}{\PYZhy{}} \PY{l+m+mi}{1}
        \PY{k}{while} \PY{n}{j} \PY{o}{\PYZgt{}} \PY{n}{i}\PY{p}{:}
            \PY{n}{cmp\PYZus{}ops} \PY{o}{+}\PY{o}{=} \PY{l+m+mi}{1}
            \PY{k}{if} \PY{n}{a}\PY{p}{[}\PY{n}{j}\PY{p}{]} \PY{o}{\PYZgt{}} \PY{n}{a}\PY{p}{[}\PY{n}{j} \PY{o}{\PYZhy{}} \PY{l+m+mi}{1}\PY{p}{]}\PY{p}{:}
                \PY{n}{cpy\PYZus{}ops} \PY{o}{+}\PY{o}{=} \PY{l+m+mi}{2}
                \PY{n}{a}\PY{p}{[}\PY{n}{j}\PY{p}{]}\PY{p}{,} \PY{n}{a}\PY{p}{[}\PY{n}{j} \PY{o}{\PYZhy{}} \PY{l+m+mi}{1}\PY{p}{]} \PY{o}{=} \PY{n}{a}\PY{p}{[}\PY{n}{j} \PY{o}{\PYZhy{}} \PY{l+m+mi}{1}\PY{p}{]}\PY{p}{,} \PY{n}{a}\PY{p}{[}\PY{n}{j}\PY{p}{]}
            \PY{n}{j} \PY{o}{\PYZhy{}}\PY{o}{=} \PY{l+m+mi}{1}
        \PY{n}{i} \PY{o}{+}\PY{o}{=} \PY{l+m+mi}{1}

    \PY{k}{return} \PY{n}{cmp\PYZus{}ops}\PY{p}{,} \PY{n}{cpy\PYZus{}ops}
\end{Verbatim}
\end{tcolorbox}

    \begin{tcolorbox}[breakable, size=fbox, boxrule=1pt, pad at break*=1mm,colback=cellbackground, colframe=cellborder]
\prompt{In}{incolor}{84}{\boxspacing}
\begin{Verbatim}[commandchars=\\\{\}]
\PY{n}{a} \PY{o}{=} \PY{p}{[}\PY{n}{randint}\PY{p}{(}\PY{l+m+mi}{0}\PY{p}{,} \PY{l+m+mi}{10}\PY{p}{)} \PY{k}{for} \PY{n}{\PYZus{}} \PY{o+ow}{in} \PY{n+nb}{range}\PY{p}{(}\PY{l+m+mi}{10}\PY{p}{)}\PY{p}{]}
\PY{n+nb}{print}\PY{p}{(}\PY{l+s+sa}{f}\PY{l+s+s2}{\PYZdq{}}\PY{l+s+s2}{Unsortierte Liste:}\PY{l+s+se}{\PYZbs{}n}\PY{l+s+si}{\PYZob{}}\PY{n}{a}\PY{l+s+si}{\PYZcb{}}\PY{l+s+s2}{\PYZdq{}}\PY{p}{)}

\PY{n}{cmp}\PY{p}{,} \PY{n}{cpy} \PY{o}{=} \PY{n}{bubble\PYZus{}sort}\PY{p}{(}\PY{n}{a}\PY{p}{)}
\PY{n+nb}{print}\PY{p}{(}\PY{l+s+sa}{f}\PY{l+s+s2}{\PYZdq{}}\PY{l+s+s2}{Sortierte Liste:}\PY{l+s+se}{\PYZbs{}n}\PY{l+s+si}{\PYZob{}}\PY{n}{a}\PY{l+s+si}{\PYZcb{}}\PY{l+s+s2}{\PYZdq{}}\PY{p}{)}
\PY{n+nb}{print}\PY{p}{(}\PY{l+s+sa}{f}\PY{l+s+s2}{\PYZdq{}}\PY{l+s+s2}{Vergleichs\PYZhy{} \PYZam{} Kopieroperationen:}\PY{l+s+se}{\PYZbs{}n}\PY{l+s+si}{\PYZob{}}\PY{n}{cmp}\PY{l+s+si}{\PYZcb{}}\PY{l+s+s2}{, }\PY{l+s+si}{\PYZob{}}\PY{n}{cpy}\PY{l+s+si}{\PYZcb{}}\PY{l+s+s2}{\PYZdq{}}\PY{p}{)}
\end{Verbatim}
\end{tcolorbox}

    \begin{Verbatim}[commandchars=\\\{\}]
Unsortierte Liste:
[2, 0, 8, 3, 4, 3, 8, 6, 0, 6]
Sortierte Liste:
[8, 8, 6, 6, 4, 3, 3, 2, 0, 0]
Vergleichs- \& Kopieroperationen:
45, 50
    \end{Verbatim}

    Implementieren Sie die Functionen \emph{\texttt{selection\_sort()}} und
\emph{\texttt{insertion\_sort()}} in den unten stehenden Zellen.

\hypertarget{a-selection_sort---4p.}{%
\subsection{\texorpdfstring{a) \emph{selection\_sort()} -
4P.}{a) selection\_sort() - 4P.}}\label{a-selection_sort---4p.}}

Die Function \emph{selection\_sort()} soll den Selection-Sort
Algorithmus aus der Vorlesung implementieren. Die Funktion erwartet eine
beliebige Liste als Eingabe, welche dann sortiert werden soll. Die Liste
soll absteigend sortiert werden. Zusätzlich sollen die Anzahl der
Vergleichs- und Kopieroperationen gezählt werden und am Ende der
Funktion zurückgeben.

Orientieren Sie sich bei Ihrer Implementierung an dem Pseudocode, der in
der Vorlesung vorgestellt worden ist, sowie an der Funktion
\emph{bubble\_sort()}. Bitte beachten Sie, dass es insbesondere nicht
erlaubt ist die Liste zuerst aufsteigend zu sortieren und in einem
zweiten Schritt die Reihenfolge der Zahlen zu ändern. Ihr Algorithmus
soll direkt beim Durchlauf die Zahlen absteigend sortieren. Sie dürfen
die vordefinierten Listen von Python benutzen, zusätzlich dürfen Sie die
Funktion len() verwenden. Alle anderen Funktionen dürfen Sie nicht
benutzen, so auch nicht die Funktion sort() oder reverse().

    \begin{tcolorbox}[breakable, size=fbox, boxrule=1pt, pad at break*=1mm,colback=cellbackground, colframe=cellborder]
\prompt{In}{incolor}{85}{\boxspacing}
\begin{Verbatim}[commandchars=\\\{\}]
\PY{k}{def} \PY{n+nf}{selection\PYZus{}sort}\PY{p}{(}\PY{n}{a}\PY{p}{)}\PY{p}{:}
    \PY{n}{cmp\PYZus{}ops} \PY{o}{=} \PY{l+m+mi}{0}  \PY{c+c1}{\PYZsh{} number of compare operations}
    \PY{n}{cpy\PYZus{}ops} \PY{o}{=} \PY{l+m+mi}{0}  \PY{c+c1}{\PYZsh{} number of copy operations}

    \PY{c+c1}{\PYZsh{} YOUR CODE HERE}
    \PY{n}{n} \PY{o}{=} \PY{n+nb}{len}\PY{p}{(}\PY{n}{a}\PY{p}{)}
    \PY{k}{for} \PY{n}{i} \PY{o+ow}{in} \PY{n+nb}{range}\PY{p}{(}\PY{l+m+mi}{0}\PY{p}{,}\PY{n}{n}\PY{o}{\PYZhy{}}\PY{l+m+mi}{1}\PY{p}{)}\PY{p}{:}
        \PY{n+nb}{max} \PY{o}{=} \PY{n}{i}
        \PY{k}{for} \PY{n}{j} \PY{o+ow}{in} \PY{n+nb}{range}\PY{p}{(}\PY{n}{i}\PY{o}{+}\PY{l+m+mi}{1}\PY{p}{,}\PY{n}{n}\PY{p}{)}\PY{p}{:}
            \PY{k}{if} \PY{n}{a}\PY{p}{[}\PY{n}{j}\PY{p}{]} \PY{o}{\PYZgt{}} \PY{n}{a}\PY{p}{[}\PY{n+nb}{max}\PY{p}{]}\PY{p}{:}
                \PY{n+nb}{max} \PY{o}{=} \PY{n}{j}
                \PY{n}{cmp\PYZus{}ops} \PY{o}{+}\PY{o}{=} \PY{l+m+mi}{1}
            \PY{k}{else}\PY{p}{:}
                \PY{n}{cmp\PYZus{}ops} \PY{o}{+}\PY{o}{=} \PY{l+m+mi}{1}
        \PY{k}{if} \PY{n+nb}{max} \PY{o}{!=} \PY{n}{i}\PY{p}{:}
           \PY{n}{tmp} \PY{o}{=} \PY{n}{a}\PY{p}{[}\PY{n}{i}\PY{p}{]}
           \PY{n}{a}\PY{p}{[}\PY{n}{i}\PY{p}{]} \PY{o}{=} \PY{n}{a}\PY{p}{[}\PY{n+nb}{max}\PY{p}{]}
           \PY{n}{a}\PY{p}{[}\PY{n+nb}{max}\PY{p}{]} \PY{o}{=} \PY{n}{tmp}
           \PY{n}{cpy\PYZus{}ops} \PY{o}{+}\PY{o}{=} \PY{l+m+mi}{2}
                
        
    \PY{c+c1}{\PYZsh{} raise NotImplementedError()}
    \PY{k}{return} \PY{n}{cmp\PYZus{}ops}\PY{p}{,} \PY{n}{cpy\PYZus{}ops}
\end{Verbatim}
\end{tcolorbox}

    \hypertarget{a---test-cases}{%
\subsection{a) - Test Cases}\label{a---test-cases}}

    \begin{tcolorbox}[breakable, size=fbox, boxrule=1pt, pad at break*=1mm,colback=cellbackground, colframe=cellborder]
\prompt{In}{incolor}{86}{\boxspacing}
\begin{Verbatim}[commandchars=\\\{\}]
\PY{c+c1}{\PYZsh{} public unittests}
\PY{c+c1}{\PYZsh{} test if list is sorted after function call}
\PY{n}{l\PYZus{}unsorted} \PY{o}{=} \PY{p}{[}\PY{l+m+mi}{0}\PY{p}{,} \PY{l+m+mi}{1}\PY{p}{,} \PY{l+m+mi}{2}\PY{p}{,} \PY{l+m+mi}{3}\PY{p}{,} \PY{l+m+mi}{4}\PY{p}{,} \PY{l+m+mi}{5}\PY{p}{,} \PY{l+m+mi}{6}\PY{p}{,} \PY{l+m+mi}{7}\PY{p}{,} \PY{l+m+mi}{8}\PY{p}{,} \PY{l+m+mi}{9}\PY{p}{]}
\PY{n}{l\PYZus{}sorted} \PY{o}{=} \PY{p}{[}\PY{l+m+mi}{9}\PY{p}{,} \PY{l+m+mi}{8}\PY{p}{,} \PY{l+m+mi}{7}\PY{p}{,} \PY{l+m+mi}{6}\PY{p}{,} \PY{l+m+mi}{5}\PY{p}{,} \PY{l+m+mi}{4}\PY{p}{,} \PY{l+m+mi}{3}\PY{p}{,} \PY{l+m+mi}{2}\PY{p}{,} \PY{l+m+mi}{1}\PY{p}{,} \PY{l+m+mi}{0}\PY{p}{]}
\PY{n}{selection\PYZus{}sort}\PY{p}{(}\PY{n}{l\PYZus{}unsorted}\PY{p}{)}
\PY{n}{assert\PYZus{}equal}\PY{p}{(}\PY{n+nb}{str}\PY{p}{(}\PY{n}{l\PYZus{}unsorted}\PY{p}{)}\PY{p}{,} \PY{n+nb}{str}\PY{p}{(}\PY{n}{l\PYZus{}sorted}\PY{p}{)}\PY{p}{)}

\PY{n}{l\PYZus{}unsorted} \PY{o}{=} \PY{p}{[}\PY{l+s+s1}{\PYZsq{}}\PY{l+s+s1}{asd}\PY{l+s+s1}{\PYZsq{}}\PY{p}{,} \PY{l+s+s1}{\PYZsq{}}\PY{l+s+s1}{qwe}\PY{l+s+s1}{\PYZsq{}}\PY{p}{,} \PY{l+s+s1}{\PYZsq{}}\PY{l+s+s1}{fgh}\PY{l+s+s1}{\PYZsq{}}\PY{p}{,} \PY{l+s+s1}{\PYZsq{}}\PY{l+s+s1}{iop}\PY{l+s+s1}{\PYZsq{}}\PY{p}{]}
\PY{n}{l\PYZus{}sorted} \PY{o}{=} \PY{p}{[}\PY{l+s+s1}{\PYZsq{}}\PY{l+s+s1}{qwe}\PY{l+s+s1}{\PYZsq{}}\PY{p}{,} \PY{l+s+s1}{\PYZsq{}}\PY{l+s+s1}{iop}\PY{l+s+s1}{\PYZsq{}}\PY{p}{,} \PY{l+s+s1}{\PYZsq{}}\PY{l+s+s1}{fgh}\PY{l+s+s1}{\PYZsq{}}\PY{p}{,} \PY{l+s+s1}{\PYZsq{}}\PY{l+s+s1}{asd}\PY{l+s+s1}{\PYZsq{}}\PY{p}{]}
\PY{n}{selection\PYZus{}sort}\PY{p}{(}\PY{n}{l\PYZus{}unsorted}\PY{p}{)}
\PY{n}{assert\PYZus{}equal}\PY{p}{(}\PY{n+nb}{str}\PY{p}{(}\PY{n}{l\PYZus{}unsorted}\PY{p}{)}\PY{p}{,} \PY{n+nb}{str}\PY{p}{(}\PY{n}{l\PYZus{}sorted}\PY{p}{)}\PY{p}{)}

\PY{c+c1}{\PYZsh{}check yourself if the list is ordered in a descending order}
\PY{n}{a} \PY{o}{=} \PY{p}{[}\PY{n}{randint}\PY{p}{(}\PY{l+m+mi}{0}\PY{p}{,} \PY{l+m+mi}{10}\PY{p}{)} \PY{k}{for} \PY{n}{\PYZus{}} \PY{o+ow}{in} \PY{n+nb}{range}\PY{p}{(}\PY{l+m+mi}{10}\PY{p}{)}\PY{p}{]}
\PY{n+nb}{print}\PY{p}{(}\PY{l+s+sa}{f}\PY{l+s+s2}{\PYZdq{}}\PY{l+s+s2}{Unsortierte Liste:}\PY{l+s+se}{\PYZbs{}n}\PY{l+s+si}{\PYZob{}}\PY{n}{a}\PY{l+s+si}{\PYZcb{}}\PY{l+s+s2}{\PYZdq{}}\PY{p}{)}

\PY{n}{cmp}\PY{p}{,} \PY{n}{cpy} \PY{o}{=} \PY{n}{selection\PYZus{}sort}\PY{p}{(}\PY{n}{a}\PY{p}{)}
\PY{n+nb}{print}\PY{p}{(}\PY{l+s+sa}{f}\PY{l+s+s2}{\PYZdq{}}\PY{l+s+s2}{Sortierte Liste:}\PY{l+s+se}{\PYZbs{}n}\PY{l+s+si}{\PYZob{}}\PY{n}{a}\PY{l+s+si}{\PYZcb{}}\PY{l+s+s2}{\PYZdq{}}\PY{p}{)}
\PY{n+nb}{print}\PY{p}{(}\PY{l+s+sa}{f}\PY{l+s+s2}{\PYZdq{}}\PY{l+s+s2}{Vergleichs\PYZhy{} \PYZam{} Kopieroperationen:}\PY{l+s+se}{\PYZbs{}n}\PY{l+s+si}{\PYZob{}}\PY{n}{cmp}\PY{l+s+si}{\PYZcb{}}\PY{l+s+s2}{, }\PY{l+s+si}{\PYZob{}}\PY{n}{cpy}\PY{l+s+si}{\PYZcb{}}\PY{l+s+s2}{\PYZdq{}}\PY{p}{)}
\end{Verbatim}
\end{tcolorbox}

    \begin{Verbatim}[commandchars=\\\{\}]
Unsortierte Liste:
[3, 9, 3, 9, 5, 10, 9, 10, 5, 5]
Sortierte Liste:
[10, 10, 9, 9, 9, 5, 5, 5, 3, 3]
Vergleichs- \& Kopieroperationen:
45, 16
    \end{Verbatim}

    \hypertarget{b-insertion_sort---4p.}{%
\subsection{\texorpdfstring{b) \emph{insertion\_sort()} -
4P.}{b) insertion\_sort() - 4P.}}\label{b-insertion_sort---4p.}}

Die Function \emph{insertion\_sort()} soll den Insertion-Sort
Algorithmus aus der Vorlesung implementieren, allerdings soll die Liste
absteigend sortiert werden. Sie bekommen eine sortierte oder unsortierte
Liste als Eingabe und sollen diese sortieren. Zusätzlich sollen Sie die
Anzahl der Vergleichs- und Kopieroperationen in ihrem Algorithmus zählen
und am Ende der Funktion zurückgeben.

Bitte orientieren Sie sich in Ihrer Implementierung an dem Pseudocode,
der in der Vorlesung vorgestellt worden ist, sowie an der Funktion
\emph{bubble\_sort()}. Bitte beachten Sie zusätzlich, dass es
insbesondere nicht erlaubt ist die Liste zuerst aufsteigend zu sortieren
und in einem zweiten Schritt die Reihenfolge der Zahlen zu ändern. Ihr
Algorithmus soll direkt beim Durchlauf die Zahlen absteigend
einsortieren.

    \begin{tcolorbox}[breakable, size=fbox, boxrule=1pt, pad at break*=1mm,colback=cellbackground, colframe=cellborder]
\prompt{In}{incolor}{87}{\boxspacing}
\begin{Verbatim}[commandchars=\\\{\}]
\PY{k}{def} \PY{n+nf}{insertion\PYZus{}sort}\PY{p}{(}\PY{n}{a}\PY{p}{)}\PY{p}{:}
    \PY{n}{cmp\PYZus{}ops} \PY{o}{=} \PY{l+m+mi}{0}  \PY{c+c1}{\PYZsh{} number of compare operations}
    \PY{n}{cpy\PYZus{}ops} \PY{o}{=} \PY{l+m+mi}{0}  \PY{c+c1}{\PYZsh{} number of copy operations}

    \PY{c+c1}{\PYZsh{} YOUR CODE HERE}
    \PY{n}{n} \PY{o}{=} \PY{n+nb}{len}\PY{p}{(}\PY{n}{a}\PY{p}{)}
    \PY{k}{for} \PY{n}{i} \PY{o+ow}{in} \PY{n+nb}{range}\PY{p}{(}\PY{l+m+mi}{1}\PY{p}{,}\PY{n}{n}\PY{p}{)}\PY{p}{:}
        \PY{n}{key} \PY{o}{=} \PY{n}{a}\PY{p}{[}\PY{n}{i}\PY{p}{]}
        \PY{n}{j}\PY{o}{=} \PY{n}{i}\PY{o}{\PYZhy{}}\PY{l+m+mi}{1}
        \PY{n}{cmp\PYZus{}ops} \PY{o}{+}\PY{o}{=} \PY{l+m+mi}{1}
        \PY{k}{while} \PY{n}{j}\PY{o}{\PYZgt{}}\PY{o}{=} \PY{l+m+mi}{0} \PY{o+ow}{and} \PY{n}{key} \PY{o}{\PYZgt{}} \PY{n}{a}\PY{p}{[}\PY{n}{j}\PY{p}{]}\PY{p}{:}
            \PY{n}{a}\PY{p}{[}\PY{n}{j} \PY{o}{+} \PY{l+m+mi}{1}\PY{p}{]} \PY{o}{=} \PY{n}{a}\PY{p}{[}\PY{n}{j}\PY{p}{]}
            \PY{n}{cpy\PYZus{}ops} \PY{o}{+}\PY{o}{=} \PY{l+m+mi}{1}
            \PY{n}{j} \PY{o}{\PYZhy{}}\PY{o}{=} \PY{l+m+mi}{1}
        \PY{k}{if} \PY{n}{i} \PY{o}{!=} \PY{n}{j} \PY{o}{+} \PY{l+m+mi}{1}\PY{p}{:}
            \PY{n}{a}\PY{p}{[}\PY{n}{j} \PY{o}{+} \PY{l+m+mi}{1}\PY{p}{]} \PY{o}{=} \PY{n}{key}
            \PY{n}{cpy\PYZus{}ops} \PY{o}{+}\PY{o}{=} \PY{l+m+mi}{1}
        

    \PY{c+c1}{\PYZsh{} raise NotImplementedError()}
    \PY{k}{return} \PY{n}{cmp\PYZus{}ops}\PY{p}{,} \PY{n}{cpy\PYZus{}ops}
\end{Verbatim}
\end{tcolorbox}

    \hypertarget{b---test-cases}{%
\subsection{b) - Test Cases}\label{b---test-cases}}

    \begin{tcolorbox}[breakable, size=fbox, boxrule=1pt, pad at break*=1mm,colback=cellbackground, colframe=cellborder]
\prompt{In}{incolor}{88}{\boxspacing}
\begin{Verbatim}[commandchars=\\\{\}]
\PY{c+c1}{\PYZsh{} public unittests}
\PY{c+c1}{\PYZsh{} test if list is sorted after function call}
\PY{n}{l\PYZus{}unsorted} \PY{o}{=} \PY{p}{[}\PY{l+m+mi}{0}\PY{p}{,} \PY{l+m+mi}{1}\PY{p}{,} \PY{l+m+mi}{2}\PY{p}{,} \PY{l+m+mi}{3}\PY{p}{,} \PY{l+m+mi}{4}\PY{p}{,} \PY{l+m+mi}{5}\PY{p}{,} \PY{l+m+mi}{6}\PY{p}{,} \PY{l+m+mi}{7}\PY{p}{,} \PY{l+m+mi}{8}\PY{p}{,} \PY{l+m+mi}{9}\PY{p}{]}
\PY{n}{l\PYZus{}sorted} \PY{o}{=} \PY{p}{[}\PY{l+m+mi}{9}\PY{p}{,} \PY{l+m+mi}{8}\PY{p}{,} \PY{l+m+mi}{7}\PY{p}{,} \PY{l+m+mi}{6}\PY{p}{,} \PY{l+m+mi}{5}\PY{p}{,} \PY{l+m+mi}{4}\PY{p}{,} \PY{l+m+mi}{3}\PY{p}{,} \PY{l+m+mi}{2}\PY{p}{,} \PY{l+m+mi}{1}\PY{p}{,} \PY{l+m+mi}{0}\PY{p}{]}
\PY{n}{insertion\PYZus{}sort}\PY{p}{(}\PY{n}{l\PYZus{}unsorted}\PY{p}{)}
\PY{n+nb}{print}\PY{p}{(}\PY{n}{l\PYZus{}unsorted}\PY{p}{)}
\PY{n}{assert\PYZus{}equal}\PY{p}{(}\PY{n+nb}{str}\PY{p}{(}\PY{n}{l\PYZus{}unsorted}\PY{p}{)}\PY{p}{,} \PY{n+nb}{str}\PY{p}{(}\PY{n}{l\PYZus{}sorted}\PY{p}{)}\PY{p}{)}

\PY{n}{l\PYZus{}unsorted} \PY{o}{=} \PY{p}{[}\PY{l+s+s1}{\PYZsq{}}\PY{l+s+s1}{asd}\PY{l+s+s1}{\PYZsq{}}\PY{p}{,} \PY{l+s+s1}{\PYZsq{}}\PY{l+s+s1}{qwe}\PY{l+s+s1}{\PYZsq{}}\PY{p}{,} \PY{l+s+s1}{\PYZsq{}}\PY{l+s+s1}{fgh}\PY{l+s+s1}{\PYZsq{}}\PY{p}{,} \PY{l+s+s1}{\PYZsq{}}\PY{l+s+s1}{iop}\PY{l+s+s1}{\PYZsq{}}\PY{p}{]}
\PY{n}{l\PYZus{}sorted} \PY{o}{=} \PY{p}{[}\PY{l+s+s1}{\PYZsq{}}\PY{l+s+s1}{qwe}\PY{l+s+s1}{\PYZsq{}}\PY{p}{,} \PY{l+s+s1}{\PYZsq{}}\PY{l+s+s1}{iop}\PY{l+s+s1}{\PYZsq{}}\PY{p}{,} \PY{l+s+s1}{\PYZsq{}}\PY{l+s+s1}{fgh}\PY{l+s+s1}{\PYZsq{}}\PY{p}{,} \PY{l+s+s1}{\PYZsq{}}\PY{l+s+s1}{asd}\PY{l+s+s1}{\PYZsq{}}\PY{p}{]}
\PY{n}{insertion\PYZus{}sort}\PY{p}{(}\PY{n}{l\PYZus{}unsorted}\PY{p}{)}
\PY{n+nb}{print}\PY{p}{(}\PY{n}{l\PYZus{}unsorted}\PY{p}{)}
\PY{n}{assert\PYZus{}equal}\PY{p}{(}\PY{n+nb}{str}\PY{p}{(}\PY{n}{l\PYZus{}unsorted}\PY{p}{)}\PY{p}{,} \PY{n+nb}{str}\PY{p}{(}\PY{n}{l\PYZus{}sorted}\PY{p}{)}\PY{p}{)}


\PY{c+c1}{\PYZsh{}check yourself if the list is ordered in a descending order}
\PY{n}{a} \PY{o}{=} \PY{p}{[}\PY{n}{randint}\PY{p}{(}\PY{l+m+mi}{0}\PY{p}{,} \PY{l+m+mi}{10}\PY{p}{)} \PY{k}{for} \PY{n}{\PYZus{}} \PY{o+ow}{in} \PY{n+nb}{range}\PY{p}{(}\PY{l+m+mi}{10}\PY{p}{)}\PY{p}{]}
\PY{n+nb}{print}\PY{p}{(}\PY{l+s+sa}{f}\PY{l+s+s2}{\PYZdq{}}\PY{l+s+s2}{Unsortierte Liste:}\PY{l+s+se}{\PYZbs{}n}\PY{l+s+si}{\PYZob{}}\PY{n}{a}\PY{l+s+si}{\PYZcb{}}\PY{l+s+s2}{\PYZdq{}}\PY{p}{)}

\PY{n}{cmp}\PY{p}{,} \PY{n}{cpy} \PY{o}{=} \PY{n}{insertion\PYZus{}sort}\PY{p}{(}\PY{n}{a}\PY{p}{)}
\PY{n+nb}{print}\PY{p}{(}\PY{l+s+sa}{f}\PY{l+s+s2}{\PYZdq{}}\PY{l+s+s2}{Sortierte Liste:}\PY{l+s+se}{\PYZbs{}n}\PY{l+s+si}{\PYZob{}}\PY{n}{a}\PY{l+s+si}{\PYZcb{}}\PY{l+s+s2}{\PYZdq{}}\PY{p}{)}
\PY{n+nb}{print}\PY{p}{(}\PY{l+s+sa}{f}\PY{l+s+s2}{\PYZdq{}}\PY{l+s+s2}{Vergleichs\PYZhy{} \PYZam{} Kopieroperationen:}\PY{l+s+se}{\PYZbs{}n}\PY{l+s+si}{\PYZob{}}\PY{n}{cmp}\PY{l+s+si}{\PYZcb{}}\PY{l+s+s2}{, }\PY{l+s+si}{\PYZob{}}\PY{n}{cpy}\PY{l+s+si}{\PYZcb{}}\PY{l+s+s2}{\PYZdq{}}\PY{p}{)}
\end{Verbatim}
\end{tcolorbox}

    \begin{Verbatim}[commandchars=\\\{\}]
[9, 8, 7, 6, 5, 4, 3, 2, 1, 0]
['qwe', 'iop', 'fgh', 'asd']
Unsortierte Liste:
[1, 10, 8, 3, 4, 1, 8, 5, 8, 4]
Sortierte Liste:
[10, 8, 8, 8, 5, 4, 4, 3, 1, 1]
Vergleichs- \& Kopieroperationen:
9, 29
    \end{Verbatim}

    \hypertarget{c-vergleich-von-operationen---2p.}{%
\subsection{c) Vergleich von Operationen -
2P.}\label{c-vergleich-von-operationen---2p.}}

Als nächstes sollen Sie die in diesem Notebook implementierten Verfahren
gegeneinander testen. Der unten stehende Code führt die Sortierverfahren
auf zwei generierten Listen und berechnet zusätzlich die
Ausführungszeit. Diese Zeit ist abhängig von der verwendeten Hardware
und kann auf unterschiedlichen Umgebungen variieren. Die Anzahl der
Vergleichs- und Kopieroperationen bleibt allerdings gleich.

Führen Sie den unten stehenden Code aus und beschreiben sie kurz (3-5
Sätze) die Ergebnisse, die Sie in den Tabellen sehen. Verhalten sich die
Sortieralgorithmen so, wie in der Vorlesung beschrieben? Worin besteht
Ihrer Meinung nach der unterschied zwischen \texttt{Array\ 1} und
\texttt{Array\ 2}? Können Sie diese Ergebnisse bei mehrfacher Ausführung
von dem Code reproduzieren oder ändern sich die Ergebnisse jedes Mal?

    \begin{tcolorbox}[breakable, size=fbox, boxrule=1pt, pad at break*=1mm,colback=cellbackground, colframe=cellborder]
\prompt{In}{incolor}{89}{\boxspacing}
\begin{Verbatim}[commandchars=\\\{\}]
\PY{k}{def} \PY{n+nf}{testSort}\PY{p}{(}\PY{n}{a}\PY{p}{,} \PY{n}{sort\PYZus{}fct}\PY{p}{)}\PY{p}{:}
    \PY{n}{b} \PY{o}{=} \PY{p}{[}\PY{n}{x} \PY{k}{for} \PY{n}{x} \PY{o+ow}{in} \PY{n}{a}\PY{p}{]}
    \PY{n}{start} \PY{o}{=} \PY{n}{time}\PY{p}{(}\PY{p}{)}
    \PY{n}{cmp\PYZus{}ops}\PY{p}{,} \PY{n}{cpy\PYZus{}ops} \PY{o}{=} \PY{n}{sort\PYZus{}fct}\PY{p}{(}\PY{n}{b}\PY{p}{)}
    \PY{k}{return} \PY{n}{time}\PY{p}{(}\PY{p}{)} \PY{o}{\PYZhy{}} \PY{n}{start}\PY{p}{,} \PY{n}{cmp\PYZus{}ops}\PY{p}{,} \PY{n}{cpy\PYZus{}ops}

\PY{k}{def} \PY{n+nf}{plotRuntimeAndOperations}\PY{p}{(}\PY{n}{array}\PY{p}{,} \PY{n}{title}\PY{o}{=}\PY{k+kc}{None}\PY{p}{)}\PY{p}{:}
    \PY{n}{df} \PY{o}{=} \PY{n}{pd}\PY{o}{.}\PY{n}{DataFrame}\PY{p}{(}\PY{n}{columns}\PY{o}{=}\PY{p}{[}\PY{l+s+s1}{\PYZsq{}}\PY{l+s+s1}{Runtime}\PY{l+s+s1}{\PYZsq{}}\PY{p}{,} \PY{l+s+s1}{\PYZsq{}}\PY{l+s+s1}{Compare Operations}\PY{l+s+s1}{\PYZsq{}}\PY{p}{,} \PY{l+s+s1}{\PYZsq{}}\PY{l+s+s1}{Copy Operations}\PY{l+s+s1}{\PYZsq{}}\PY{p}{]}\PY{p}{)}
    \PY{n}{df}\PY{o}{.}\PY{n}{loc}\PY{p}{[}\PY{l+s+s1}{\PYZsq{}}\PY{l+s+s1}{Selection\PYZhy{}Sort}\PY{l+s+s1}{\PYZsq{}}\PY{p}{]} \PY{o}{=} \PY{n}{testSort}\PY{p}{(}\PY{n}{array}\PY{p}{,} \PY{n}{selection\PYZus{}sort}\PY{p}{)}
    \PY{n}{df}\PY{o}{.}\PY{n}{loc}\PY{p}{[}\PY{l+s+s1}{\PYZsq{}}\PY{l+s+s1}{Bubble\PYZhy{}Sort}\PY{l+s+s1}{\PYZsq{}}\PY{p}{]} \PY{o}{=} \PY{n}{testSort}\PY{p}{(}\PY{n}{array}\PY{p}{,} \PY{n}{bubble\PYZus{}sort}\PY{p}{)}
    \PY{n}{df}\PY{o}{.}\PY{n}{loc}\PY{p}{[}\PY{l+s+s1}{\PYZsq{}}\PY{l+s+s1}{Insertion\PYZhy{}Sort}\PY{l+s+s1}{\PYZsq{}}\PY{p}{]} \PY{o}{=} \PY{n}{testSort}\PY{p}{(}\PY{n}{array}\PY{p}{,} \PY{n}{insertion\PYZus{}sort}\PY{p}{)}
    \PY{n}{ax} \PY{o}{=} \PY{n}{df}\PY{o}{.}\PY{n}{plot}\PY{o}{.}\PY{n}{bar}\PY{p}{(}\PY{n}{title}\PY{o}{=}\PY{n}{title}\PY{p}{,} \PY{n}{rot}\PY{o}{=}\PY{l+m+mi}{0}\PY{p}{,} \PY{n}{subplots}\PY{o}{=}\PY{k+kc}{True}\PY{p}{,} \PY{n}{layout}\PY{o}{=}\PY{p}{(}\PY{l+m+mi}{1}\PY{p}{,}\PY{l+m+mi}{3}\PY{p}{)}\PY{p}{,} \PY{n}{figsize}\PY{o}{=}\PY{p}{(}\PY{l+m+mi}{15}\PY{p}{,}\PY{l+m+mi}{4}\PY{p}{)}\PY{p}{)}

\PY{n}{array\PYZus{}1} \PY{o}{=} \PY{p}{[}\PY{n}{randint}\PY{p}{(}\PY{l+m+mi}{0}\PY{p}{,} \PY{l+m+mi}{100}\PY{p}{)} \PY{k}{for} \PY{n}{\PYZus{}} \PY{o+ow}{in} \PY{n+nb}{range}\PY{p}{(}\PY{l+m+mi}{100}\PY{p}{)}\PY{p}{]}
\PY{n}{array\PYZus{}2} \PY{o}{=} \PY{n+nb}{list}\PY{p}{(}\PY{n+nb}{range}\PY{p}{(}\PY{l+m+mi}{100}\PY{p}{,} \PY{l+m+mi}{0}\PY{p}{,} \PY{o}{\PYZhy{}}\PY{l+m+mi}{1}\PY{p}{)}\PY{p}{)}

\PY{n}{plotRuntimeAndOperations}\PY{p}{(}\PY{n}{array\PYZus{}1}\PY{p}{,} \PY{l+s+s2}{\PYZdq{}}\PY{l+s+s2}{Array 1}\PY{l+s+s2}{\PYZdq{}}\PY{p}{)}
\PY{n}{plotRuntimeAndOperations}\PY{p}{(}\PY{n}{array\PYZus{}2}\PY{p}{,} \PY{l+s+s2}{\PYZdq{}}\PY{l+s+s2}{Array 2}\PY{l+s+s2}{\PYZdq{}}\PY{p}{)}
\end{Verbatim}
\end{tcolorbox}

    \begin{center}
    \adjustimage{max size={0.9\linewidth}{0.9\paperheight}}{blatt04-python_files/blatt04-python_21_0.png}
    \end{center}
    { \hspace*{\fill} \\}
    
    \begin{center}
    \adjustimage{max size={0.9\linewidth}{0.9\paperheight}}{blatt04-python_files/blatt04-python_21_1.png}
    \end{center}
    { \hspace*{\fill} \\}
    
    die Sortieralgorithmen verhalten sich vie im Vorlesung nut insertion
sort ist anders weil es absteigend sortieren muss, die unterschide
zwichen beide arrays liegt daran dass Array 1 randomiziert ist und Array
2 nicht randomiziert ist, deswegen ist das runtime von array1
unterschiedlich nach jeder ausfuhrung von dieser code block.

    \hypertarget{jupyter-notebook-stolperfalle}{%
\subsection{Jupyter Notebook
Stolperfalle}\label{jupyter-notebook-stolperfalle}}

Bei der Benutzung von Jupyter Notebooks, wird der globale Zustand aller
Variablen zwischen der Ausführung von verschiedenen Zellen erhalten.
Dies ist auch der Fall, wenn man Zellen löscht oder hinzufügt. Um sicher
zu gehen, dass Sie nicht ausversehen notwendige Variablen überschrieben
oder gelöscht haben benutzen Sie bitte
\texttt{Kernel\ -\textgreater{}\ Restart\ \&\ Run\ All}.

Zudem möchten wir Sie bitten zu überprüfen, ob alle öffentlichen
Unittests bestanden werden. Dies stellt sicher, dass wir ihren Code mit
unserem automatisierten System bewerten können.


    % Add a bibliography block to the postdoc
    
    
    
\end{document}
