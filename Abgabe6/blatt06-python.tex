\documentclass[11pt]{article}

    \usepackage[breakable]{tcolorbox}
    \usepackage{parskip} % Stop auto-indenting (to mimic markdown behaviour)
    

    % Basic figure setup, for now with no caption control since it's done
    % automatically by Pandoc (which extracts ![](path) syntax from Markdown).
    \usepackage{graphicx}
    % Maintain compatibility with old templates. Remove in nbconvert 6.0
    \let\Oldincludegraphics\includegraphics
    % Ensure that by default, figures have no caption (until we provide a
    % proper Figure object with a Caption API and a way to capture that
    % in the conversion process - todo).
    \usepackage{caption}
    \DeclareCaptionFormat{nocaption}{}
    \captionsetup{format=nocaption,aboveskip=0pt,belowskip=0pt}

    \usepackage{float}
    \floatplacement{figure}{H} % forces figures to be placed at the correct location
    \usepackage{xcolor} % Allow colors to be defined
    \usepackage{enumerate} % Needed for markdown enumerations to work
    \usepackage{geometry} % Used to adjust the document margins
    \usepackage{amsmath} % Equations
    \usepackage{amssymb} % Equations
    \usepackage{textcomp} % defines textquotesingle
    % Hack from http://tex.stackexchange.com/a/47451/13684:
    \AtBeginDocument{%
        \def\PYZsq{\textquotesingle}% Upright quotes in Pygmentized code
    }
    \usepackage{upquote} % Upright quotes for verbatim code
    \usepackage{eurosym} % defines \euro

    \usepackage{iftex}
    \ifPDFTeX
        \usepackage[T1]{fontenc}
        \IfFileExists{alphabeta.sty}{
              \usepackage{alphabeta}
          }{
              \usepackage[mathletters]{ucs}
              \usepackage[utf8x]{inputenc}
          }
    \else
        \usepackage{fontspec}
        \usepackage{unicode-math}
    \fi

    \usepackage{fancyvrb} % verbatim replacement that allows latex
    \usepackage{grffile} % extends the file name processing of package graphics
                         % to support a larger range
    \makeatletter % fix for old versions of grffile with XeLaTeX
    \@ifpackagelater{grffile}{2019/11/01}
    {
      % Do nothing on new versions
    }
    {
      \def\Gread@@xetex#1{%
        \IfFileExists{"\Gin@base".bb}%
        {\Gread@eps{\Gin@base.bb}}%
        {\Gread@@xetex@aux#1}%
      }
    }
    \makeatother
    \usepackage[Export]{adjustbox} % Used to constrain images to a maximum size
    \adjustboxset{max size={0.9\linewidth}{0.9\paperheight}}

    % The hyperref package gives us a pdf with properly built
    % internal navigation ('pdf bookmarks' for the table of contents,
    % internal cross-reference links, web links for URLs, etc.)
    \usepackage{hyperref}
    % The default LaTeX title has an obnoxious amount of whitespace. By default,
    % titling removes some of it. It also provides customization options.
    \usepackage{titling}
    \usepackage{longtable} % longtable support required by pandoc >1.10
    \usepackage{booktabs}  % table support for pandoc > 1.12.2
    \usepackage{array}     % table support for pandoc >= 2.11.3
    \usepackage{calc}      % table minipage width calculation for pandoc >= 2.11.1
    \usepackage[inline]{enumitem} % IRkernel/repr support (it uses the enumerate* environment)
    \usepackage[normalem]{ulem} % ulem is needed to support strikethroughs (\sout)
                                % normalem makes italics be italics, not underlines
    \usepackage{mathrsfs}
    

    
    % Colors for the hyperref package
    \definecolor{urlcolor}{rgb}{0,.145,.698}
    \definecolor{linkcolor}{rgb}{.71,0.21,0.01}
    \definecolor{citecolor}{rgb}{.12,.54,.11}

    % ANSI colors
    \definecolor{ansi-black}{HTML}{3E424D}
    \definecolor{ansi-black-intense}{HTML}{282C36}
    \definecolor{ansi-red}{HTML}{E75C58}
    \definecolor{ansi-red-intense}{HTML}{B22B31}
    \definecolor{ansi-green}{HTML}{00A250}
    \definecolor{ansi-green-intense}{HTML}{007427}
    \definecolor{ansi-yellow}{HTML}{DDB62B}
    \definecolor{ansi-yellow-intense}{HTML}{B27D12}
    \definecolor{ansi-blue}{HTML}{208FFB}
    \definecolor{ansi-blue-intense}{HTML}{0065CA}
    \definecolor{ansi-magenta}{HTML}{D160C4}
    \definecolor{ansi-magenta-intense}{HTML}{A03196}
    \definecolor{ansi-cyan}{HTML}{60C6C8}
    \definecolor{ansi-cyan-intense}{HTML}{258F8F}
    \definecolor{ansi-white}{HTML}{C5C1B4}
    \definecolor{ansi-white-intense}{HTML}{A1A6B2}
    \definecolor{ansi-default-inverse-fg}{HTML}{FFFFFF}
    \definecolor{ansi-default-inverse-bg}{HTML}{000000}

    % common color for the border for error outputs.
    \definecolor{outerrorbackground}{HTML}{FFDFDF}

    % commands and environments needed by pandoc snippets
    % extracted from the output of `pandoc -s`
    \providecommand{\tightlist}{%
      \setlength{\itemsep}{0pt}\setlength{\parskip}{0pt}}
    \DefineVerbatimEnvironment{Highlighting}{Verbatim}{commandchars=\\\{\}}
    % Add ',fontsize=\small' for more characters per line
    \newenvironment{Shaded}{}{}
    \newcommand{\KeywordTok}[1]{\textcolor[rgb]{0.00,0.44,0.13}{\textbf{{#1}}}}
    \newcommand{\DataTypeTok}[1]{\textcolor[rgb]{0.56,0.13,0.00}{{#1}}}
    \newcommand{\DecValTok}[1]{\textcolor[rgb]{0.25,0.63,0.44}{{#1}}}
    \newcommand{\BaseNTok}[1]{\textcolor[rgb]{0.25,0.63,0.44}{{#1}}}
    \newcommand{\FloatTok}[1]{\textcolor[rgb]{0.25,0.63,0.44}{{#1}}}
    \newcommand{\CharTok}[1]{\textcolor[rgb]{0.25,0.44,0.63}{{#1}}}
    \newcommand{\StringTok}[1]{\textcolor[rgb]{0.25,0.44,0.63}{{#1}}}
    \newcommand{\CommentTok}[1]{\textcolor[rgb]{0.38,0.63,0.69}{\textit{{#1}}}}
    \newcommand{\OtherTok}[1]{\textcolor[rgb]{0.00,0.44,0.13}{{#1}}}
    \newcommand{\AlertTok}[1]{\textcolor[rgb]{1.00,0.00,0.00}{\textbf{{#1}}}}
    \newcommand{\FunctionTok}[1]{\textcolor[rgb]{0.02,0.16,0.49}{{#1}}}
    \newcommand{\RegionMarkerTok}[1]{{#1}}
    \newcommand{\ErrorTok}[1]{\textcolor[rgb]{1.00,0.00,0.00}{\textbf{{#1}}}}
    \newcommand{\NormalTok}[1]{{#1}}

    % Additional commands for more recent versions of Pandoc
    \newcommand{\ConstantTok}[1]{\textcolor[rgb]{0.53,0.00,0.00}{{#1}}}
    \newcommand{\SpecialCharTok}[1]{\textcolor[rgb]{0.25,0.44,0.63}{{#1}}}
    \newcommand{\VerbatimStringTok}[1]{\textcolor[rgb]{0.25,0.44,0.63}{{#1}}}
    \newcommand{\SpecialStringTok}[1]{\textcolor[rgb]{0.73,0.40,0.53}{{#1}}}
    \newcommand{\ImportTok}[1]{{#1}}
    \newcommand{\DocumentationTok}[1]{\textcolor[rgb]{0.73,0.13,0.13}{\textit{{#1}}}}
    \newcommand{\AnnotationTok}[1]{\textcolor[rgb]{0.38,0.63,0.69}{\textbf{\textit{{#1}}}}}
    \newcommand{\CommentVarTok}[1]{\textcolor[rgb]{0.38,0.63,0.69}{\textbf{\textit{{#1}}}}}
    \newcommand{\VariableTok}[1]{\textcolor[rgb]{0.10,0.09,0.49}{{#1}}}
    \newcommand{\ControlFlowTok}[1]{\textcolor[rgb]{0.00,0.44,0.13}{\textbf{{#1}}}}
    \newcommand{\OperatorTok}[1]{\textcolor[rgb]{0.40,0.40,0.40}{{#1}}}
    \newcommand{\BuiltInTok}[1]{{#1}}
    \newcommand{\ExtensionTok}[1]{{#1}}
    \newcommand{\PreprocessorTok}[1]{\textcolor[rgb]{0.74,0.48,0.00}{{#1}}}
    \newcommand{\AttributeTok}[1]{\textcolor[rgb]{0.49,0.56,0.16}{{#1}}}
    \newcommand{\InformationTok}[1]{\textcolor[rgb]{0.38,0.63,0.69}{\textbf{\textit{{#1}}}}}
    \newcommand{\WarningTok}[1]{\textcolor[rgb]{0.38,0.63,0.69}{\textbf{\textit{{#1}}}}}


    % Define a nice break command that doesn't care if a line doesn't already
    % exist.
    \def\br{\hspace*{\fill} \\* }
    % Math Jax compatibility definitions
    \def\gt{>}
    \def\lt{<}
    \let\Oldtex\TeX
    \let\Oldlatex\LaTeX
    \renewcommand{\TeX}{\textrm{\Oldtex}}
    \renewcommand{\LaTeX}{\textrm{\Oldlatex}}
    % Document parameters
    % Document title
    \title{blatt06-python}
    
    
    
    
    
% Pygments definitions
\makeatletter
\def\PY@reset{\let\PY@it=\relax \let\PY@bf=\relax%
    \let\PY@ul=\relax \let\PY@tc=\relax%
    \let\PY@bc=\relax \let\PY@ff=\relax}
\def\PY@tok#1{\csname PY@tok@#1\endcsname}
\def\PY@toks#1+{\ifx\relax#1\empty\else%
    \PY@tok{#1}\expandafter\PY@toks\fi}
\def\PY@do#1{\PY@bc{\PY@tc{\PY@ul{%
    \PY@it{\PY@bf{\PY@ff{#1}}}}}}}
\def\PY#1#2{\PY@reset\PY@toks#1+\relax+\PY@do{#2}}

\@namedef{PY@tok@w}{\def\PY@tc##1{\textcolor[rgb]{0.73,0.73,0.73}{##1}}}
\@namedef{PY@tok@c}{\let\PY@it=\textit\def\PY@tc##1{\textcolor[rgb]{0.24,0.48,0.48}{##1}}}
\@namedef{PY@tok@cp}{\def\PY@tc##1{\textcolor[rgb]{0.61,0.40,0.00}{##1}}}
\@namedef{PY@tok@k}{\let\PY@bf=\textbf\def\PY@tc##1{\textcolor[rgb]{0.00,0.50,0.00}{##1}}}
\@namedef{PY@tok@kp}{\def\PY@tc##1{\textcolor[rgb]{0.00,0.50,0.00}{##1}}}
\@namedef{PY@tok@kt}{\def\PY@tc##1{\textcolor[rgb]{0.69,0.00,0.25}{##1}}}
\@namedef{PY@tok@o}{\def\PY@tc##1{\textcolor[rgb]{0.40,0.40,0.40}{##1}}}
\@namedef{PY@tok@ow}{\let\PY@bf=\textbf\def\PY@tc##1{\textcolor[rgb]{0.67,0.13,1.00}{##1}}}
\@namedef{PY@tok@nb}{\def\PY@tc##1{\textcolor[rgb]{0.00,0.50,0.00}{##1}}}
\@namedef{PY@tok@nf}{\def\PY@tc##1{\textcolor[rgb]{0.00,0.00,1.00}{##1}}}
\@namedef{PY@tok@nc}{\let\PY@bf=\textbf\def\PY@tc##1{\textcolor[rgb]{0.00,0.00,1.00}{##1}}}
\@namedef{PY@tok@nn}{\let\PY@bf=\textbf\def\PY@tc##1{\textcolor[rgb]{0.00,0.00,1.00}{##1}}}
\@namedef{PY@tok@ne}{\let\PY@bf=\textbf\def\PY@tc##1{\textcolor[rgb]{0.80,0.25,0.22}{##1}}}
\@namedef{PY@tok@nv}{\def\PY@tc##1{\textcolor[rgb]{0.10,0.09,0.49}{##1}}}
\@namedef{PY@tok@no}{\def\PY@tc##1{\textcolor[rgb]{0.53,0.00,0.00}{##1}}}
\@namedef{PY@tok@nl}{\def\PY@tc##1{\textcolor[rgb]{0.46,0.46,0.00}{##1}}}
\@namedef{PY@tok@ni}{\let\PY@bf=\textbf\def\PY@tc##1{\textcolor[rgb]{0.44,0.44,0.44}{##1}}}
\@namedef{PY@tok@na}{\def\PY@tc##1{\textcolor[rgb]{0.41,0.47,0.13}{##1}}}
\@namedef{PY@tok@nt}{\let\PY@bf=\textbf\def\PY@tc##1{\textcolor[rgb]{0.00,0.50,0.00}{##1}}}
\@namedef{PY@tok@nd}{\def\PY@tc##1{\textcolor[rgb]{0.67,0.13,1.00}{##1}}}
\@namedef{PY@tok@s}{\def\PY@tc##1{\textcolor[rgb]{0.73,0.13,0.13}{##1}}}
\@namedef{PY@tok@sd}{\let\PY@it=\textit\def\PY@tc##1{\textcolor[rgb]{0.73,0.13,0.13}{##1}}}
\@namedef{PY@tok@si}{\let\PY@bf=\textbf\def\PY@tc##1{\textcolor[rgb]{0.64,0.35,0.47}{##1}}}
\@namedef{PY@tok@se}{\let\PY@bf=\textbf\def\PY@tc##1{\textcolor[rgb]{0.67,0.36,0.12}{##1}}}
\@namedef{PY@tok@sr}{\def\PY@tc##1{\textcolor[rgb]{0.64,0.35,0.47}{##1}}}
\@namedef{PY@tok@ss}{\def\PY@tc##1{\textcolor[rgb]{0.10,0.09,0.49}{##1}}}
\@namedef{PY@tok@sx}{\def\PY@tc##1{\textcolor[rgb]{0.00,0.50,0.00}{##1}}}
\@namedef{PY@tok@m}{\def\PY@tc##1{\textcolor[rgb]{0.40,0.40,0.40}{##1}}}
\@namedef{PY@tok@gh}{\let\PY@bf=\textbf\def\PY@tc##1{\textcolor[rgb]{0.00,0.00,0.50}{##1}}}
\@namedef{PY@tok@gu}{\let\PY@bf=\textbf\def\PY@tc##1{\textcolor[rgb]{0.50,0.00,0.50}{##1}}}
\@namedef{PY@tok@gd}{\def\PY@tc##1{\textcolor[rgb]{0.63,0.00,0.00}{##1}}}
\@namedef{PY@tok@gi}{\def\PY@tc##1{\textcolor[rgb]{0.00,0.52,0.00}{##1}}}
\@namedef{PY@tok@gr}{\def\PY@tc##1{\textcolor[rgb]{0.89,0.00,0.00}{##1}}}
\@namedef{PY@tok@ge}{\let\PY@it=\textit}
\@namedef{PY@tok@gs}{\let\PY@bf=\textbf}
\@namedef{PY@tok@gp}{\let\PY@bf=\textbf\def\PY@tc##1{\textcolor[rgb]{0.00,0.00,0.50}{##1}}}
\@namedef{PY@tok@go}{\def\PY@tc##1{\textcolor[rgb]{0.44,0.44,0.44}{##1}}}
\@namedef{PY@tok@gt}{\def\PY@tc##1{\textcolor[rgb]{0.00,0.27,0.87}{##1}}}
\@namedef{PY@tok@err}{\def\PY@bc##1{{\setlength{\fboxsep}{\string -\fboxrule}\fcolorbox[rgb]{1.00,0.00,0.00}{1,1,1}{\strut ##1}}}}
\@namedef{PY@tok@kc}{\let\PY@bf=\textbf\def\PY@tc##1{\textcolor[rgb]{0.00,0.50,0.00}{##1}}}
\@namedef{PY@tok@kd}{\let\PY@bf=\textbf\def\PY@tc##1{\textcolor[rgb]{0.00,0.50,0.00}{##1}}}
\@namedef{PY@tok@kn}{\let\PY@bf=\textbf\def\PY@tc##1{\textcolor[rgb]{0.00,0.50,0.00}{##1}}}
\@namedef{PY@tok@kr}{\let\PY@bf=\textbf\def\PY@tc##1{\textcolor[rgb]{0.00,0.50,0.00}{##1}}}
\@namedef{PY@tok@bp}{\def\PY@tc##1{\textcolor[rgb]{0.00,0.50,0.00}{##1}}}
\@namedef{PY@tok@fm}{\def\PY@tc##1{\textcolor[rgb]{0.00,0.00,1.00}{##1}}}
\@namedef{PY@tok@vc}{\def\PY@tc##1{\textcolor[rgb]{0.10,0.09,0.49}{##1}}}
\@namedef{PY@tok@vg}{\def\PY@tc##1{\textcolor[rgb]{0.10,0.09,0.49}{##1}}}
\@namedef{PY@tok@vi}{\def\PY@tc##1{\textcolor[rgb]{0.10,0.09,0.49}{##1}}}
\@namedef{PY@tok@vm}{\def\PY@tc##1{\textcolor[rgb]{0.10,0.09,0.49}{##1}}}
\@namedef{PY@tok@sa}{\def\PY@tc##1{\textcolor[rgb]{0.73,0.13,0.13}{##1}}}
\@namedef{PY@tok@sb}{\def\PY@tc##1{\textcolor[rgb]{0.73,0.13,0.13}{##1}}}
\@namedef{PY@tok@sc}{\def\PY@tc##1{\textcolor[rgb]{0.73,0.13,0.13}{##1}}}
\@namedef{PY@tok@dl}{\def\PY@tc##1{\textcolor[rgb]{0.73,0.13,0.13}{##1}}}
\@namedef{PY@tok@s2}{\def\PY@tc##1{\textcolor[rgb]{0.73,0.13,0.13}{##1}}}
\@namedef{PY@tok@sh}{\def\PY@tc##1{\textcolor[rgb]{0.73,0.13,0.13}{##1}}}
\@namedef{PY@tok@s1}{\def\PY@tc##1{\textcolor[rgb]{0.73,0.13,0.13}{##1}}}
\@namedef{PY@tok@mb}{\def\PY@tc##1{\textcolor[rgb]{0.40,0.40,0.40}{##1}}}
\@namedef{PY@tok@mf}{\def\PY@tc##1{\textcolor[rgb]{0.40,0.40,0.40}{##1}}}
\@namedef{PY@tok@mh}{\def\PY@tc##1{\textcolor[rgb]{0.40,0.40,0.40}{##1}}}
\@namedef{PY@tok@mi}{\def\PY@tc##1{\textcolor[rgb]{0.40,0.40,0.40}{##1}}}
\@namedef{PY@tok@il}{\def\PY@tc##1{\textcolor[rgb]{0.40,0.40,0.40}{##1}}}
\@namedef{PY@tok@mo}{\def\PY@tc##1{\textcolor[rgb]{0.40,0.40,0.40}{##1}}}
\@namedef{PY@tok@ch}{\let\PY@it=\textit\def\PY@tc##1{\textcolor[rgb]{0.24,0.48,0.48}{##1}}}
\@namedef{PY@tok@cm}{\let\PY@it=\textit\def\PY@tc##1{\textcolor[rgb]{0.24,0.48,0.48}{##1}}}
\@namedef{PY@tok@cpf}{\let\PY@it=\textit\def\PY@tc##1{\textcolor[rgb]{0.24,0.48,0.48}{##1}}}
\@namedef{PY@tok@c1}{\let\PY@it=\textit\def\PY@tc##1{\textcolor[rgb]{0.24,0.48,0.48}{##1}}}
\@namedef{PY@tok@cs}{\let\PY@it=\textit\def\PY@tc##1{\textcolor[rgb]{0.24,0.48,0.48}{##1}}}

\def\PYZbs{\char`\\}
\def\PYZus{\char`\_}
\def\PYZob{\char`\{}
\def\PYZcb{\char`\}}
\def\PYZca{\char`\^}
\def\PYZam{\char`\&}
\def\PYZlt{\char`\<}
\def\PYZgt{\char`\>}
\def\PYZsh{\char`\#}
\def\PYZpc{\char`\%}
\def\PYZdl{\char`\$}
\def\PYZhy{\char`\-}
\def\PYZsq{\char`\'}
\def\PYZdq{\char`\"}
\def\PYZti{\char`\~}
% for compatibility with earlier versions
\def\PYZat{@}
\def\PYZlb{[}
\def\PYZrb{]}
\makeatother


    % For linebreaks inside Verbatim environment from package fancyvrb.
    \makeatletter
        \newbox\Wrappedcontinuationbox
        \newbox\Wrappedvisiblespacebox
        \newcommand*\Wrappedvisiblespace {\textcolor{red}{\textvisiblespace}}
        \newcommand*\Wrappedcontinuationsymbol {\textcolor{red}{\llap{\tiny$\m@th\hookrightarrow$}}}
        \newcommand*\Wrappedcontinuationindent {3ex }
        \newcommand*\Wrappedafterbreak {\kern\Wrappedcontinuationindent\copy\Wrappedcontinuationbox}
        % Take advantage of the already applied Pygments mark-up to insert
        % potential linebreaks for TeX processing.
        %        {, <, #, %, $, ' and ": go to next line.
        %        _, }, ^, &, >, - and ~: stay at end of broken line.
        % Use of \textquotesingle for straight quote.
        \newcommand*\Wrappedbreaksatspecials {%
            \def\PYGZus{\discretionary{\char`\_}{\Wrappedafterbreak}{\char`\_}}%
            \def\PYGZob{\discretionary{}{\Wrappedafterbreak\char`\{}{\char`\{}}%
            \def\PYGZcb{\discretionary{\char`\}}{\Wrappedafterbreak}{\char`\}}}%
            \def\PYGZca{\discretionary{\char`\^}{\Wrappedafterbreak}{\char`\^}}%
            \def\PYGZam{\discretionary{\char`\&}{\Wrappedafterbreak}{\char`\&}}%
            \def\PYGZlt{\discretionary{}{\Wrappedafterbreak\char`\<}{\char`\<}}%
            \def\PYGZgt{\discretionary{\char`\>}{\Wrappedafterbreak}{\char`\>}}%
            \def\PYGZsh{\discretionary{}{\Wrappedafterbreak\char`\#}{\char`\#}}%
            \def\PYGZpc{\discretionary{}{\Wrappedafterbreak\char`\%}{\char`\%}}%
            \def\PYGZdl{\discretionary{}{\Wrappedafterbreak\char`\$}{\char`\$}}%
            \def\PYGZhy{\discretionary{\char`\-}{\Wrappedafterbreak}{\char`\-}}%
            \def\PYGZsq{\discretionary{}{\Wrappedafterbreak\textquotesingle}{\textquotesingle}}%
            \def\PYGZdq{\discretionary{}{\Wrappedafterbreak\char`\"}{\char`\"}}%
            \def\PYGZti{\discretionary{\char`\~}{\Wrappedafterbreak}{\char`\~}}%
        }
        % Some characters . , ; ? ! / are not pygmentized.
        % This macro makes them "active" and they will insert potential linebreaks
        \newcommand*\Wrappedbreaksatpunct {%
            \lccode`\~`\.\lowercase{\def~}{\discretionary{\hbox{\char`\.}}{\Wrappedafterbreak}{\hbox{\char`\.}}}%
            \lccode`\~`\,\lowercase{\def~}{\discretionary{\hbox{\char`\,}}{\Wrappedafterbreak}{\hbox{\char`\,}}}%
            \lccode`\~`\;\lowercase{\def~}{\discretionary{\hbox{\char`\;}}{\Wrappedafterbreak}{\hbox{\char`\;}}}%
            \lccode`\~`\:\lowercase{\def~}{\discretionary{\hbox{\char`\:}}{\Wrappedafterbreak}{\hbox{\char`\:}}}%
            \lccode`\~`\?\lowercase{\def~}{\discretionary{\hbox{\char`\?}}{\Wrappedafterbreak}{\hbox{\char`\?}}}%
            \lccode`\~`\!\lowercase{\def~}{\discretionary{\hbox{\char`\!}}{\Wrappedafterbreak}{\hbox{\char`\!}}}%
            \lccode`\~`\/\lowercase{\def~}{\discretionary{\hbox{\char`\/}}{\Wrappedafterbreak}{\hbox{\char`\/}}}%
            \catcode`\.\active
            \catcode`\,\active
            \catcode`\;\active
            \catcode`\:\active
            \catcode`\?\active
            \catcode`\!\active
            \catcode`\/\active
            \lccode`\~`\~
        }
    \makeatother

    \let\OriginalVerbatim=\Verbatim
    \makeatletter
    \renewcommand{\Verbatim}[1][1]{%
        %\parskip\z@skip
        \sbox\Wrappedcontinuationbox {\Wrappedcontinuationsymbol}%
        \sbox\Wrappedvisiblespacebox {\FV@SetupFont\Wrappedvisiblespace}%
        \def\FancyVerbFormatLine ##1{\hsize\linewidth
            \vtop{\raggedright\hyphenpenalty\z@\exhyphenpenalty\z@
                \doublehyphendemerits\z@\finalhyphendemerits\z@
                \strut ##1\strut}%
        }%
        % If the linebreak is at a space, the latter will be displayed as visible
        % space at end of first line, and a continuation symbol starts next line.
        % Stretch/shrink are however usually zero for typewriter font.
        \def\FV@Space {%
            \nobreak\hskip\z@ plus\fontdimen3\font minus\fontdimen4\font
            \discretionary{\copy\Wrappedvisiblespacebox}{\Wrappedafterbreak}
            {\kern\fontdimen2\font}%
        }%

        % Allow breaks at special characters using \PYG... macros.
        \Wrappedbreaksatspecials
        % Breaks at punctuation characters . , ; ? ! and / need catcode=\active
        \OriginalVerbatim[#1,codes*=\Wrappedbreaksatpunct]%
    }
    \makeatother

    % Exact colors from NB
    \definecolor{incolor}{HTML}{303F9F}
    \definecolor{outcolor}{HTML}{D84315}
    \definecolor{cellborder}{HTML}{CFCFCF}
    \definecolor{cellbackground}{HTML}{F7F7F7}

    % prompt
    \makeatletter
    \newcommand{\boxspacing}{\kern\kvtcb@left@rule\kern\kvtcb@boxsep}
    \makeatother
    \newcommand{\prompt}[4]{
        {\ttfamily\llap{{\color{#2}[#3]:\hspace{3pt}#4}}\vspace{-\baselineskip}}
    }
    

    
    % Prevent overflowing lines due to hard-to-break entities
    \sloppy
    % Setup hyperref package
    \hypersetup{
      breaklinks=true,  % so long urls are correctly broken across lines
      colorlinks=true,
      urlcolor=urlcolor,
      linkcolor=linkcolor,
      citecolor=citecolor,
      }
    % Slightly bigger margins than the latex defaults
    
    \geometry{verbose,tmargin=1in,bmargin=1in,lmargin=1in,rmargin=1in}
    
    

\begin{document}
    
    \maketitle
    
    

    
    \hypertarget{datenstrukturen-und-algorithmen}{%
\section{Datenstrukturen und
Algorithmen}\label{datenstrukturen-und-algorithmen}}

\hypertarget{praktische-aufgabe-3}{%
\subsection{Praktische Aufgabe 3}\label{praktische-aufgabe-3}}

In dieser praktischen Aufgabe werden Sie sich mit binären Suchbäumen
(BST = binary search tree) beschäftigen. Zunächst implementieren Sie die
aus der Vorlesung bekannte Funktion \texttt{insert()}, die jetzt neben
einem Schlüssel auch einen Wert im Binärbaum (BT = binary tree)
abspeichert. Danach überprüfen Sie, ob ein gegebener BT auch ein BST ist
und falls nicht, implementieren Sie einen einfachen Algorithmus, um
jeden beliebiegen BT in einen BST mit einer vorgegebenen Baumstruktur zu
überführen.

Die Abgaben werden mit der \texttt{nbgrader} Erweiterung korrigiert. Das
System erwartet, dass der Code zum Lösen der Aufgaben nach der
\texttt{\#YOUR\ CODE\ HERE} Anweisung kommt. Außerdem darf die
Zellenreihenfolge nicht geändert werden. Damit Sie selbst Ihre
Lösungsvorschläge validieren können, werden Ihnen Unittests zur
Verfügung gestellt. Beachten Sie, dass diese Tests keine Garantie sind
für das Erreichen der vollen Punktzahl, da Sie nur einen Teil der
Funktionalität überprüfen.

Wichtig: Füllen Sie zunächst die erste Zelle mit
\texttt{\#YOUR\ ANSWER\ HERE} unter dem Titel \texttt{Abgabeteam} mit
ihren Namen und Matrikelnummern vollständig aus. Dies ermöglicht uns
auch bei technischen Problemen die Abgaben eindeutig zuordnen zu können.
Ändern Sie außerdem nicht den Namen der Datei.

Zusammenfassung der Aufgaben:

\begin{enumerate}
\def\labelenumi{\arabic{enumi}.}
\tightlist
\item
  \textbf{Binäre Suchbäume} - insgesamt: 20 Punkte

  \begin{itemize}
  \tightlist
  \item
    insert() - 4P.
  \item
    is\_bin\_search\_tree() - 6P.
  \item
    bin\_tree\_2\_list() - 3P.
  \item
    list\_2\_bin\_tree() - 3P.
  \item
    bin\_tree\_2\_bin\_search\_tree() - 4P.
  \end{itemize}
\end{enumerate}

    \hypertarget{abgabeteam}{%
\subsection{Abgabeteam}\label{abgabeteam}}

Bitte füllen Sie die untenstehende Zelle aus mit

Nummer des Tutoriums,

Voranme Nachname Matrikelnummer 1,

Vorname Nachname Matrikelnummer 2,

(Vorname Nachname Matrikelnummer 3)

    23,

Mohammed Al-Laktah 419664,

Salah Atallah 414867,

    \hypertarget{module-importieren}{%
\subsection{Module importieren}\label{module-importieren}}

Als erstes importieren wir die verwendeten Module. Sie dürfen keine
weiteren Python Module importieren oder verwenden, als die hier
spezifizierten Module.

Falls Sie eine Entwicklungsumgebung, wie Google Colab oder Deepnote
verwendeen und bestimmte Module nicht verfügbar sind, kommentieren Sie
die erste Zeile aus, um die Module in der Laufzeitumgebung temporär zu
installieren.

    \begin{tcolorbox}[breakable, size=fbox, boxrule=1pt, pad at break*=1mm,colback=cellbackground, colframe=cellborder]
\prompt{In}{incolor}{1}{\boxspacing}
\begin{Verbatim}[commandchars=\\\{\}]
\PY{c+ch}{\PYZsh{}!pip install nose}

\PY{k+kn}{from} \PY{n+nn}{nose}\PY{n+nn}{.}\PY{n+nn}{tools} \PY{k+kn}{import} \PY{n}{assert\PYZus{}equal}
\end{Verbatim}
\end{tcolorbox}

    \hypertarget{binuxe4re-suchbuxe4ume}{%
\section{Binäre Suchbäume}\label{binuxe4re-suchbuxe4ume}}

Im folgenden werden Sie einen BST implementieren, der sowohl ein
Attribut \texttt{key}, als auch \texttt{value} besitzt. Dabei wird die
Sortierung der Blätter über den Schlüssel \texttt{key} bestimmt, während
das eigentliche Element in \texttt{value} gespeichert wird. Dadurch
können wir verschiedene Elemente nach einem beliebigen Schlüssel im Baum
speichern. Der unten stehende Code definiert die Struktur von einem
Node. Die Funktion \texttt{print\_tree} gibt jeweils den BT als einen
formattierten Output aus. Bitte beachten Sie, dass wir in dieser
Implementierung keinen Parent-Pointer nutzen. Alle Aufgaben lassen sich
auch ohne diese Referenz effizient lösen.

    \begin{tcolorbox}[breakable, size=fbox, boxrule=1pt, pad at break*=1mm,colback=cellbackground, colframe=cellborder]
\prompt{In}{incolor}{2}{\boxspacing}
\begin{Verbatim}[commandchars=\\\{\}]
\PY{k}{class} \PY{n+nc}{Node}\PY{p}{:}
    \PY{k}{def} \PY{n+nf+fm}{\PYZus{}\PYZus{}init\PYZus{}\PYZus{}}\PY{p}{(}\PY{n+nb+bp}{self}\PY{p}{,} \PY{n}{key}\PY{p}{,} \PY{n}{value}\PY{o}{=}\PY{k+kc}{None}\PY{p}{,} \PY{n}{left}\PY{o}{=}\PY{k+kc}{None}\PY{p}{,} \PY{n}{right}\PY{o}{=}\PY{k+kc}{None}\PY{p}{)}\PY{p}{:}
        \PY{n+nb+bp}{self}\PY{o}{.}\PY{n}{key} \PY{o}{=} \PY{n}{key}
        \PY{n+nb+bp}{self}\PY{o}{.}\PY{n}{value} \PY{o}{=} \PY{n}{value}
        \PY{n+nb+bp}{self}\PY{o}{.}\PY{n}{left} \PY{o}{=} \PY{n}{left}
        \PY{n+nb+bp}{self}\PY{o}{.}\PY{n}{right} \PY{o}{=} \PY{n}{right}
        

    \PY{k}{def} \PY{n+nf+fm}{\PYZus{}\PYZus{}repr\PYZus{}\PYZus{}}\PY{p}{(}\PY{n+nb+bp}{self}\PY{p}{)}\PY{p}{:}
        \PY{k}{return} \PY{l+s+sa}{f}\PY{l+s+s2}{\PYZdq{}}\PY{l+s+s2}{(}\PY{l+s+si}{\PYZob{}}\PY{n+nb+bp}{self}\PY{o}{.}\PY{n}{left}\PY{l+s+si}{\PYZcb{}}\PY{l+s+s2}{, }\PY{l+s+si}{\PYZob{}}\PY{n+nb+bp}{self}\PY{o}{.}\PY{n}{key}\PY{l+s+si}{\PYZcb{}}\PY{l+s+s2}{|}\PY{l+s+si}{\PYZob{}}\PY{n+nb+bp}{self}\PY{o}{.}\PY{n}{value}\PY{l+s+si}{\PYZcb{}}\PY{l+s+s2}{, }\PY{l+s+si}{\PYZob{}}\PY{n+nb+bp}{self}\PY{o}{.}\PY{n}{right}\PY{l+s+si}{\PYZcb{}}\PY{l+s+s2}{)}\PY{l+s+s2}{\PYZdq{}}


\PY{k}{def} \PY{n+nf}{print\PYZus{}tree}\PY{p}{(}\PY{n}{node}\PY{p}{,} \PY{n}{level}\PY{o}{=}\PY{l+m+mi}{0}\PY{p}{)}\PY{p}{:}
    \PY{k}{if} \PY{n}{node} \PY{o+ow}{is} \PY{o+ow}{not} \PY{k+kc}{None}\PY{p}{:}
        \PY{n}{print\PYZus{}tree}\PY{p}{(}\PY{n}{node}\PY{o}{.}\PY{n}{right}\PY{p}{,} \PY{n}{level} \PY{o}{+} \PY{l+m+mi}{1}\PY{p}{)}
        \PY{n+nb}{print}\PY{p}{(}\PY{l+s+sa}{f}\PY{l+s+s2}{\PYZdq{}}\PY{l+s+si}{\PYZob{}}\PY{l+s+s1}{\PYZsq{}}\PY{l+s+s1}{ }\PY{l+s+s1}{\PYZsq{}} \PY{o}{*} \PY{l+m+mi}{8} \PY{o}{*} \PY{n}{level}\PY{l+s+si}{\PYZcb{}}\PY{l+s+s2}{\PYZhy{}\PYZgt{}  }\PY{l+s+si}{\PYZob{}}\PY{n+nb}{str}\PY{p}{(}\PY{n}{node}\PY{o}{.}\PY{n}{key}\PY{p}{)}\PY{l+s+si}{\PYZcb{}}\PY{l+s+s2}{|}\PY{l+s+si}{\PYZob{}}\PY{n+nb}{str}\PY{p}{(}\PY{n}{node}\PY{o}{.}\PY{n}{value}\PY{p}{)}\PY{l+s+si}{\PYZcb{}}\PY{l+s+s2}{\PYZdq{}}\PY{p}{)}
        \PY{n}{print\PYZus{}tree}\PY{p}{(}\PY{n}{node}\PY{o}{.}\PY{n}{left}\PY{p}{,} \PY{n}{level} \PY{o}{+} \PY{l+m+mi}{1}\PY{p}{)}
\end{Verbatim}
\end{tcolorbox}

    \begin{tcolorbox}[breakable, size=fbox, boxrule=1pt, pad at break*=1mm,colback=cellbackground, colframe=cellborder]
\prompt{In}{incolor}{3}{\boxspacing}
\begin{Verbatim}[commandchars=\\\{\}]
\PY{n+nb}{print}\PY{p}{(}\PY{l+s+s2}{\PYZdq{}}\PY{l+s+s2}{Valid BST:}\PY{l+s+s2}{\PYZdq{}}\PY{p}{)}
\PY{n}{valid\PYZus{}bst} \PY{o}{=} \PY{n}{Node}\PY{p}{(}\PY{l+m+mi}{4}\PY{p}{,} \PY{l+m+mi}{421}\PY{p}{,} \PY{n}{Node}\PY{p}{(}\PY{l+m+mi}{2}\PY{p}{,} \PY{l+m+mi}{123}\PY{p}{,} \PY{n}{Node}\PY{p}{(}\PY{l+m+mi}{1}\PY{p}{,} \PY{l+s+s2}{\PYZdq{}}\PY{l+s+s2}{asd}\PY{l+s+s2}{\PYZdq{}}\PY{p}{)}\PY{p}{,} \PY{n}{Node}\PY{p}{(}\PY{l+m+mi}{3}\PY{p}{,} \PY{k+kc}{None}\PY{p}{,} \PY{n}{right}\PY{o}{=}\PY{n}{Node}\PY{p}{(}\PY{l+m+mi}{3}\PY{p}{,} \PY{l+s+s2}{\PYZdq{}}\PY{l+s+s2}{vwa}\PY{l+s+s2}{\PYZdq{}}\PY{p}{)}\PY{p}{)}\PY{p}{)}\PY{p}{,} \PY{n}{Node}\PY{p}{(}\PY{l+m+mi}{6}\PY{p}{,} \PY{l+m+mi}{234}\PY{p}{,} \PY{n}{Node}\PY{p}{(}\PY{l+m+mi}{5}\PY{p}{,} \PY{l+s+s2}{\PYZdq{}}\PY{l+s+s2}{abc}\PY{l+s+s2}{\PYZdq{}}\PY{p}{)}\PY{p}{,} \PY{n}{Node}\PY{p}{(}\PY{l+m+mi}{7}\PY{p}{,} \PY{l+s+s2}{\PYZdq{}}\PY{l+s+s2}{abv}\PY{l+s+s2}{\PYZdq{}}\PY{p}{,} \PY{n}{right}\PY{o}{=}\PY{n}{Node}\PY{p}{(}\PY{l+m+mi}{8}\PY{p}{,} \PY{l+s+s2}{\PYZdq{}}\PY{l+s+s2}{cde}\PY{l+s+s2}{\PYZdq{}}\PY{p}{)}\PY{p}{)}\PY{p}{)}\PY{p}{)}
\PY{n}{print\PYZus{}tree}\PY{p}{(}\PY{n}{valid\PYZus{}bst}\PY{p}{)}
\PY{n+nb}{print}\PY{p}{(}\PY{n}{valid\PYZus{}bst}\PY{p}{)}

\PY{n+nb}{print}\PY{p}{(}\PY{l+s+s2}{\PYZdq{}}\PY{l+s+se}{\PYZbs{}n}\PY{l+s+s2}{\PYZhy{}\PYZhy{}\PYZhy{}\PYZhy{}\PYZhy{}\PYZhy{}\PYZhy{}\PYZhy{}\PYZhy{}\PYZhy{}\PYZhy{}\PYZhy{}\PYZhy{}\PYZhy{}\PYZhy{}\PYZhy{}\PYZhy{}\PYZhy{}}\PY{l+s+se}{\PYZbs{}n}\PY{l+s+s2}{\PYZdq{}}\PY{p}{)}

\PY{n+nb}{print}\PY{p}{(}\PY{l+s+s2}{\PYZdq{}}\PY{l+s+s2}{Invalid BST:}\PY{l+s+s2}{\PYZdq{}}\PY{p}{)}
\PY{n}{invalid\PYZus{}bst} \PY{o}{=} \PY{n}{Node}\PY{p}{(}\PY{l+m+mi}{1}\PY{p}{,} \PY{l+m+mi}{421}\PY{p}{,} \PY{n}{Node}\PY{p}{(}\PY{l+m+mi}{2}\PY{p}{,} \PY{l+s+s2}{\PYZdq{}}\PY{l+s+s2}{kve}\PY{l+s+s2}{\PYZdq{}}\PY{p}{,} \PY{n}{Node}\PY{p}{(}\PY{l+m+mi}{4}\PY{p}{,} \PY{l+s+s2}{\PYZdq{}}\PY{l+s+s2}{abc}\PY{l+s+s2}{\PYZdq{}}\PY{p}{,} \PY{n}{Node}\PY{p}{(}\PY{l+m+mi}{0}\PY{p}{,} \PY{l+s+s2}{\PYZdq{}}\PY{l+s+s2}{esi}\PY{l+s+s2}{\PYZdq{}}\PY{p}{)}\PY{p}{)}\PY{p}{)}\PY{p}{,} \PY{n}{Node}\PY{p}{(}\PY{l+m+mi}{9}\PY{p}{,} \PY{l+s+s2}{\PYZdq{}}\PY{l+s+s2}{agg}\PY{l+s+s2}{\PYZdq{}}\PY{p}{,} \PY{n}{right}\PY{o}{=}\PY{n}{Node}\PY{p}{(}\PY{l+m+mi}{4}\PY{p}{,} \PY{l+s+s2}{\PYZdq{}}\PY{l+s+s2}{dnf}\PY{l+s+s2}{\PYZdq{}}\PY{p}{,} \PY{n}{Node}\PY{p}{(}\PY{l+m+mi}{2}\PY{p}{,} \PY{l+m+mi}{123}\PY{p}{)}\PY{p}{)}\PY{p}{)}\PY{p}{)}
\PY{n}{print\PYZus{}tree}\PY{p}{(}\PY{n}{invalid\PYZus{}bst}\PY{p}{)}
\PY{n+nb}{print}\PY{p}{(}\PY{n}{invalid\PYZus{}bst}\PY{p}{)}
\end{Verbatim}
\end{tcolorbox}

    \begin{Verbatim}[commandchars=\\\{\}]
Valid BST:
                        ->  8|cde
                ->  7|abv
        ->  6|234
                ->  5|abc
->  4|421
                        ->  3|vwa
                ->  3|None
        ->  2|123
                ->  1|asd
(((None, 1|asd, None), 2|123, (None, 3|None, (None, 3|vwa, None))), 4|421,
((None, 5|abc, None), 6|234, (None, 7|abv, (None, 8|cde, None))))

------------------

Invalid BST:
                ->  4|dnf
                        ->  2|123
        ->  9|agg
->  1|421
        ->  2|kve
                ->  4|abc
                        ->  0|esi
((((None, 0|esi, None), 4|abc, None), 2|kve, None), 1|421, (None, 9|agg, ((None,
2|123, None), 4|dnf, None)))
    \end{Verbatim}

    \hypertarget{a-insert---4p.}{%
\subsection{a) insert() - 4P.}\label{a-insert---4p.}}

Implementieren Sie die Funktion \texttt{insert()}, die \texttt{node},
\texttt{key}, sowie \texttt{value} als Parameter übergegeben bekommt.
\texttt{node} ist hierbei der Elternknoten in den ein Element
\texttt{value} entsprechend des Schlüssels \texttt{key} eingefügt wird.
Der resultierende Baum soll dabei die Eigenschaften eines BST behalten.
Orientieren sie sich an der aus der Vorlesung bekannten Funktion
\texttt{insert()} und wandeln diese an den nötigen Stellen ab.

Wenn zweimal der gleiche \texttt{key} eingefügt wird, dann wird einfach
nur \texttt{value} ersetzt.

    \begin{tcolorbox}[breakable, size=fbox, boxrule=1pt, pad at break*=1mm,colback=cellbackground, colframe=cellborder]
\prompt{In}{incolor}{4}{\boxspacing}
\begin{Verbatim}[commandchars=\\\{\}]
\PY{k}{def} \PY{n+nf}{insert}\PY{p}{(}\PY{n}{node}\PY{p}{,} \PY{n}{key}\PY{p}{,} \PY{n}{value}\PY{p}{)}\PY{p}{:}
    \PY{l+s+sd}{\PYZdq{}\PYZdq{}\PYZdq{} Insert a new node into a binary search tree based on `key` which holds `value`. \PYZdq{}\PYZdq{}\PYZdq{}}
    \PY{k}{if} \PY{o+ow}{not} \PY{n}{node}\PY{p}{:}
        \PY{n}{node} \PY{o}{=} \PY{n}{Node}\PY{p}{(}\PY{n}{key}\PY{p}{,} \PY{n}{value}\PY{p}{)}
    \PY{k}{elif} \PY{n}{key} \PY{o}{\PYZgt{}} \PY{n}{node}\PY{o}{.}\PY{n}{key} \PY{o+ow}{and} \PY{o+ow}{not} \PY{n}{node}\PY{o}{.}\PY{n}{right}\PY{p}{:}
        \PY{n}{node}\PY{o}{.}\PY{n}{right} \PY{o}{=} \PY{n}{Node}\PY{p}{(}\PY{n}{key}\PY{p}{,} \PY{n}{value}\PY{p}{)}
    \PY{k}{elif} \PY{n}{key} \PY{o}{\PYZlt{}} \PY{n}{node}\PY{o}{.}\PY{n}{key} \PY{o+ow}{and} \PY{o+ow}{not} \PY{n}{node}\PY{o}{.}\PY{n}{left}\PY{p}{:}
        \PY{n}{node}\PY{o}{.}\PY{n}{left} \PY{o}{=} \PY{n}{Node}\PY{p}{(}\PY{n}{key}\PY{p}{,} \PY{n}{value}\PY{p}{)}
    \PY{k}{elif} \PY{n}{key} \PY{o}{\PYZgt{}} \PY{n}{node}\PY{o}{.}\PY{n}{key} \PY{o+ow}{and} \PY{n}{node}\PY{o}{.}\PY{n}{right}\PY{p}{:}
           \PY{n}{insert}\PY{p}{(}\PY{n}{node}\PY{o}{.}\PY{n}{right}\PY{p}{,} \PY{n}{key}\PY{p}{,} \PY{n}{value}\PY{p}{)}
    \PY{k}{elif} \PY{n}{key} \PY{o}{\PYZlt{}} \PY{n}{node}\PY{o}{.}\PY{n}{key} \PY{o+ow}{and} \PY{n}{node}\PY{o}{.}\PY{n}{left}\PY{p}{:}
           \PY{n}{insert}\PY{p}{(}\PY{n}{node}\PY{o}{.}\PY{n}{left}\PY{p}{,} \PY{n}{key}\PY{p}{,} \PY{n}{value}\PY{p}{)}
           
    \PY{k}{return} \PY{n}{node}
\end{Verbatim}
\end{tcolorbox}

    \hypertarget{a-tests}{%
\subsection{a) Tests}\label{a-tests}}

    \begin{tcolorbox}[breakable, size=fbox, boxrule=1pt, pad at break*=1mm,colback=cellbackground, colframe=cellborder]
\prompt{In}{incolor}{5}{\boxspacing}
\begin{Verbatim}[commandchars=\\\{\}]
\PY{n}{root} \PY{o}{=} \PY{k+kc}{None} 
\PY{n}{root} \PY{o}{=} \PY{n}{insert}\PY{p}{(}\PY{n}{root}\PY{p}{,} \PY{l+m+mi}{3}\PY{p}{,} \PY{l+m+mi}{1177}\PY{p}{)}
\PY{n}{assert\PYZus{}equal}\PY{p}{(}\PY{n+nb}{str}\PY{p}{(}\PY{n}{root}\PY{p}{)}\PY{p}{,} \PY{l+s+s2}{\PYZdq{}}\PY{l+s+s2}{(None, 3|1177, None)}\PY{l+s+s2}{\PYZdq{}}\PY{p}{)}
\end{Verbatim}
\end{tcolorbox}

    \begin{tcolorbox}[breakable, size=fbox, boxrule=1pt, pad at break*=1mm,colback=cellbackground, colframe=cellborder]
\prompt{In}{incolor}{6}{\boxspacing}
\begin{Verbatim}[commandchars=\\\{\}]
\PY{n}{root} \PY{o}{=} \PY{k+kc}{None} 
\PY{n}{root} \PY{o}{=} \PY{n}{insert}\PY{p}{(}\PY{n}{root}\PY{p}{,} \PY{l+m+mi}{0}\PY{p}{,} \PY{l+s+s2}{\PYZdq{}}\PY{l+s+s2}{Value}\PY{l+s+s2}{\PYZdq{}}\PY{p}{)}
\PY{n}{root} \PY{o}{=} \PY{n}{insert}\PY{p}{(}\PY{n}{root}\PY{p}{,} \PY{l+m+mi}{1}\PY{p}{,} \PY{l+s+s2}{\PYZdq{}}\PY{l+s+s2}{Value}\PY{l+s+s2}{\PYZdq{}}\PY{p}{)}
\PY{n}{root} \PY{o}{=} \PY{n}{insert}\PY{p}{(}\PY{n}{root}\PY{p}{,} \PY{l+m+mi}{2}\PY{p}{,} \PY{l+s+s2}{\PYZdq{}}\PY{l+s+s2}{Value}\PY{l+s+s2}{\PYZdq{}}\PY{p}{)}
\PY{n}{assert\PYZus{}equal}\PY{p}{(}\PY{n+nb}{str}\PY{p}{(}\PY{n}{root}\PY{p}{)}\PY{p}{,} \PY{l+s+s2}{\PYZdq{}}\PY{l+s+s2}{(None, 0|Value, (None, 1|Value, (None, 2|Value, None)))}\PY{l+s+s2}{\PYZdq{}}\PY{p}{)}
\end{Verbatim}
\end{tcolorbox}

    \begin{tcolorbox}[breakable, size=fbox, boxrule=1pt, pad at break*=1mm,colback=cellbackground, colframe=cellborder]
\prompt{In}{incolor}{7}{\boxspacing}
\begin{Verbatim}[commandchars=\\\{\}]
\PY{c+c1}{\PYZsh{} hidden tests}
\end{Verbatim}
\end{tcolorbox}

    Die nächste Aufgabe besteht darin einen beliebigen BT in einen BST
umzuwandeln. Dabei darf die Form des Baumes, also die Verzweigung, nicht
verändert werden.

Hierzu werden Sie zunächst eine Funktion implementieren, die überprüft
ob ein gegebener BT auch ein BST ist. Danach implementieren Sie zwei
Funktionen, die eine Inorder-Traversal über den BT durchführen und die
Nodes in eine Liste schreiben bzw. Nodes aus einer Liste in einen BT
überführen. Anschließend nutzen Sie diese Funktionen, sowie das
Sortieren von Listen, um einen beliebigen BT in einen BST umzuwandeln
und dabei die Verzweigungen des BT nicht zu verändern.

Der zu implementierende Algorithmus besteht aus den folgenden Schritten:
- Falls der BT ein BST ist: - mache nichts und geben den BT zurück -
Sonst, führe die folgenden Schritte aus: - Konvertiere den BT mit
Inorder-Traversal in eine Liste - Sortiere die Liste nach dem Schlüssel
- Konvertiere die Liste mit Inoder-Traversal in einen BT

    \hypertarget{b-is_bin_search_tree---6p.}{%
\subsection{b) is\_bin\_search\_tree() -
6P.}\label{b-is_bin_search_tree---6p.}}

Implementieren Sie die Funktion \texttt{is\_bin\_search\_tree()}, die
\texttt{node} als Parameter übergegeben bekommt und überprüft, ob es
sich um den BT der durch \texttt{node} dargestellt wird, um einen BST
handelt. Die Funktion soll \texttt{True} zurückgeben, falls ein BST
übergegeben wurde, sonst gibt sie \texttt{False} aus.

\begin{verbatim}
Eingabe:
Node(4, None, Node(2), Node(6))

Ausgabe:
True


Eingabe:
Node(3, None, Node(2,None, Node(1), Node(4)), Node(5))

Ausgabe:
False
\end{verbatim}

    \begin{tcolorbox}[breakable, size=fbox, boxrule=1pt, pad at break*=1mm,colback=cellbackground, colframe=cellborder]
\prompt{In}{incolor}{8}{\boxspacing}
\begin{Verbatim}[commandchars=\\\{\}]
\PY{k}{def} \PY{n+nf}{is\PYZus{}bin\PYZus{}search\PYZus{}tree}\PY{p}{(}\PY{n}{node}\PY{p}{)}\PY{p}{:}
    \PY{l+s+sd}{\PYZdq{}\PYZdq{}\PYZdq{} Check if a given binary tree is a binary search tree. \PYZdq{}\PYZdq{}\PYZdq{}}
    \PY{n}{isbin} \PY{o}{=} \PY{n}{node} \PY{o+ow}{is} \PY{o+ow}{not} \PY{k+kc}{None}
    
    \PY{k}{if} \PY{n}{node}\PY{o}{.}\PY{n}{right} \PY{o+ow}{and} \PY{n}{isbin}\PY{p}{:}
        \PY{n}{isbin} \PY{o}{=} \PY{n}{node}\PY{o}{.}\PY{n}{right}\PY{o}{.}\PY{n}{key} \PY{o}{\PYZgt{}} \PY{n}{node}\PY{o}{.}\PY{n}{key} \PY{o+ow}{and} \PY{n}{is\PYZus{}bin\PYZus{}search\PYZus{}tree}\PY{p}{(}\PY{n}{node}\PY{o}{.}\PY{n}{right}\PY{p}{)}
    
    \PY{k}{if} \PY{n}{node}\PY{o}{.}\PY{n}{left} \PY{o+ow}{and} \PY{n}{isbin}\PY{p}{:}
        \PY{n}{isbin} \PY{o}{=} \PY{n}{node}\PY{o}{.}\PY{n}{left}\PY{o}{.}\PY{n}{key} \PY{o}{\PYZlt{}} \PY{n}{node}\PY{o}{.}\PY{n}{key} \PY{o+ow}{and} \PY{n}{is\PYZus{}bin\PYZus{}search\PYZus{}tree}\PY{p}{(}\PY{n}{node}\PY{o}{.}\PY{n}{left}\PY{p}{)} 
    
    \PY{k}{return} \PY{n}{isbin} 
\end{Verbatim}
\end{tcolorbox}

    \hypertarget{b-tests}{%
\subsection{b) Tests}\label{b-tests}}

    \begin{tcolorbox}[breakable, size=fbox, boxrule=1pt, pad at break*=1mm,colback=cellbackground, colframe=cellborder]
\prompt{In}{incolor}{9}{\boxspacing}
\begin{Verbatim}[commandchars=\\\{\}]
\PY{c+c1}{\PYZsh{} unittests}
\PY{n}{valid\PYZus{}bst} \PY{o}{=} \PY{n}{Node}\PY{p}{(}\PY{l+m+mi}{4}\PY{p}{,} \PY{l+m+mi}{421}\PY{p}{,} \PY{n}{Node}\PY{p}{(}\PY{l+m+mi}{2}\PY{p}{,} \PY{l+m+mi}{123}\PY{p}{,} \PY{n}{Node}\PY{p}{(}\PY{l+m+mi}{1}\PY{p}{,} \PY{l+s+s2}{\PYZdq{}}\PY{l+s+s2}{asd}\PY{l+s+s2}{\PYZdq{}}\PY{p}{)}\PY{p}{,} \PY{n}{Node}\PY{p}{(}\PY{l+m+mi}{3}\PY{p}{)}\PY{p}{)}\PY{p}{,} \PY{n}{Node}\PY{p}{(}\PY{l+m+mi}{6}\PY{p}{,} \PY{l+m+mi}{234}\PY{p}{,} \PY{n}{Node}\PY{p}{(}\PY{l+m+mi}{5}\PY{p}{,} \PY{l+s+s2}{\PYZdq{}}\PY{l+s+s2}{abc}\PY{l+s+s2}{\PYZdq{}}\PY{p}{)}\PY{p}{,} \PY{n}{Node}\PY{p}{(}\PY{l+m+mi}{7}\PY{p}{,} \PY{l+s+s2}{\PYZdq{}}\PY{l+s+s2}{abv}\PY{l+s+s2}{\PYZdq{}}\PY{p}{,} \PY{n}{right}\PY{o}{=}\PY{n}{Node}\PY{p}{(}\PY{l+m+mi}{8}\PY{p}{,} \PY{l+s+s2}{\PYZdq{}}\PY{l+s+s2}{cde}\PY{l+s+s2}{\PYZdq{}}\PY{p}{)}\PY{p}{)}\PY{p}{)}\PY{p}{)}
\PY{n}{assert\PYZus{}equal}\PY{p}{(}\PY{k+kc}{True}\PY{p}{,} \PY{n}{is\PYZus{}bin\PYZus{}search\PYZus{}tree}\PY{p}{(}\PY{n}{valid\PYZus{}bst}\PY{p}{)}\PY{p}{)}

\PY{n}{invalid\PYZus{}bst} \PY{o}{=} \PY{n}{Node}\PY{p}{(}\PY{l+m+mi}{1}\PY{p}{,} \PY{l+m+mi}{421}\PY{p}{,} \PY{n}{Node}\PY{p}{(}\PY{l+m+mi}{2}\PY{p}{,} \PY{l+s+s2}{\PYZdq{}}\PY{l+s+s2}{kve}\PY{l+s+s2}{\PYZdq{}}\PY{p}{,} \PY{n}{Node}\PY{p}{(}\PY{l+m+mi}{4}\PY{p}{,} \PY{l+s+s2}{\PYZdq{}}\PY{l+s+s2}{abc}\PY{l+s+s2}{\PYZdq{}}\PY{p}{,} \PY{n}{Node}\PY{p}{(}\PY{l+m+mi}{0}\PY{p}{,} \PY{l+s+s2}{\PYZdq{}}\PY{l+s+s2}{esi}\PY{l+s+s2}{\PYZdq{}}\PY{p}{)}\PY{p}{)}\PY{p}{)}\PY{p}{,} \PY{n}{Node}\PY{p}{(}\PY{l+m+mi}{9}\PY{p}{,} \PY{l+s+s2}{\PYZdq{}}\PY{l+s+s2}{agg}\PY{l+s+s2}{\PYZdq{}}\PY{p}{,} \PY{n}{right}\PY{o}{=}\PY{n}{Node}\PY{p}{(}\PY{l+m+mi}{4}\PY{p}{,} \PY{l+s+s2}{\PYZdq{}}\PY{l+s+s2}{dnf}\PY{l+s+s2}{\PYZdq{}}\PY{p}{,} \PY{n}{Node}\PY{p}{(}\PY{l+m+mi}{2}\PY{p}{,} \PY{l+m+mi}{123}\PY{p}{)}\PY{p}{)}\PY{p}{)}\PY{p}{)}
\PY{n}{assert\PYZus{}equal}\PY{p}{(}\PY{k+kc}{False}\PY{p}{,} \PY{n}{is\PYZus{}bin\PYZus{}search\PYZus{}tree}\PY{p}{(}\PY{n}{invalid\PYZus{}bst}\PY{p}{)}\PY{p}{)}
\end{Verbatim}
\end{tcolorbox}

    \begin{tcolorbox}[breakable, size=fbox, boxrule=1pt, pad at break*=1mm,colback=cellbackground, colframe=cellborder]
\prompt{In}{incolor}{10}{\boxspacing}
\begin{Verbatim}[commandchars=\\\{\}]
\PY{c+c1}{\PYZsh{} hidden tests}
\end{Verbatim}
\end{tcolorbox}

    \begin{tcolorbox}[breakable, size=fbox, boxrule=1pt, pad at break*=1mm,colback=cellbackground, colframe=cellborder]
\prompt{In}{incolor}{11}{\boxspacing}
\begin{Verbatim}[commandchars=\\\{\}]
\PY{c+c1}{\PYZsh{} hidden tests}
\end{Verbatim}
\end{tcolorbox}

    \begin{tcolorbox}[breakable, size=fbox, boxrule=1pt, pad at break*=1mm,colback=cellbackground, colframe=cellborder]
\prompt{In}{incolor}{12}{\boxspacing}
\begin{Verbatim}[commandchars=\\\{\}]
\PY{c+c1}{\PYZsh{} hidden tests}
\end{Verbatim}
\end{tcolorbox}

    \hypertarget{c-bin_tree_2_list---3p.}{%
\subsection{c) bin\_tree\_2\_list() -
3P.}\label{c-bin_tree_2_list---3p.}}

Implementieren Sie die Funktion \texttt{bin\_tree\_2\_list()}, die einen
BT \texttt{node} erhält und eine Liste zurückgibt. Diese Liste enthält
Tupel von Schlüsseln und Werten als Elemente, die jeweils die Nodes des
BT in Inorder-Traversal beschreiben.

\begin{verbatim}
Eingabe:
Node(4, 421, Node(2, 123, Node(1, "asd"), Node(3, None, right=Node(3, "vwa"))), Node(6, 234, Node(5, "abc"), Node(7, "abv", right=Node(8, "cde"))))

Ausgabe:
[(1, 'asd'), (2, 123), (3, None), (3, 'vwa'), (4, 421), (5, 'abc'), (6, 234), (7, 'abv'), (8, 'cde')]


Eingabe:
Node(1, 421, Node(2, "kve", Node(4, "abc", Node(0, "esi"))), Node(9, "agg", right=Node(4, "dnf", Node(2, 123))))

Ausgabe:
[(0, 'esi'), (4, 'abc'), (2, 'kve'), (1, 421), (9, 'agg'), (2, 123), (4, 'dnf')]
\end{verbatim}

    \begin{tcolorbox}[breakable, size=fbox, boxrule=1pt, pad at break*=1mm,colback=cellbackground, colframe=cellborder]
\prompt{In}{incolor}{13}{\boxspacing}
\begin{Verbatim}[commandchars=\\\{\}]
\PY{k}{def} \PY{n+nf}{bin\PYZus{}tree\PYZus{}2\PYZus{}list}\PY{p}{(}\PY{n}{node}\PY{p}{)}\PY{p}{:}
    \PY{l+s+sd}{\PYZdq{}\PYZdq{}\PYZdq{} Convert keys \PYZam{} values of a binary tree into a list. \PYZdq{}\PYZdq{}\PYZdq{}}
    \PY{k}{match}\PY{p}{(}\PY{n}{node}\PY{p}{)}\PY{p}{:}
        \PY{n}{case} \PY{k+kc}{None}\PY{p}{:}
            \PY{k}{return} \PY{p}{[}\PY{p}{]}
        \PY{k}{case} \PY{k}{\PYZus{}}\PY{p}{:}
            \PY{k}{return} \PY{n}{bin\PYZus{}tree\PYZus{}2\PYZus{}list}\PY{p}{(}\PY{n}{node}\PY{o}{.}\PY{n}{left}\PY{p}{)} \PY{o}{+} \PY{p}{[}\PY{p}{(}\PY{n}{node}\PY{o}{.}\PY{n}{key}\PY{p}{,} \PY{n}{node}\PY{o}{.}\PY{n}{value}\PY{p}{)}\PY{p}{]} \PY{o}{+} \PY{n}{bin\PYZus{}tree\PYZus{}2\PYZus{}list}\PY{p}{(}\PY{n}{node}\PY{o}{.}\PY{n}{right}\PY{p}{)}
\end{Verbatim}
\end{tcolorbox}

    \hypertarget{c-tests}{%
\subsection{c) Tests}\label{c-tests}}

    \begin{tcolorbox}[breakable, size=fbox, boxrule=1pt, pad at break*=1mm,colback=cellbackground, colframe=cellborder]
\prompt{In}{incolor}{14}{\boxspacing}
\begin{Verbatim}[commandchars=\\\{\}]
\PY{n}{valid\PYZus{}bst} \PY{o}{=} \PY{n}{Node}\PY{p}{(}\PY{l+m+mi}{4}\PY{p}{,} \PY{l+m+mi}{421}\PY{p}{,} \PY{n}{Node}\PY{p}{(}\PY{l+m+mi}{2}\PY{p}{,} \PY{l+m+mi}{123}\PY{p}{,} \PY{n}{Node}\PY{p}{(}\PY{l+m+mi}{1}\PY{p}{,} \PY{l+s+s2}{\PYZdq{}}\PY{l+s+s2}{asd}\PY{l+s+s2}{\PYZdq{}}\PY{p}{)}\PY{p}{,} \PY{n}{Node}\PY{p}{(}\PY{l+m+mi}{3}\PY{p}{,} \PY{k+kc}{None}\PY{p}{,} \PY{n}{right}\PY{o}{=}\PY{n}{Node}\PY{p}{(}\PY{l+m+mi}{3}\PY{p}{,} \PY{l+s+s2}{\PYZdq{}}\PY{l+s+s2}{vwa}\PY{l+s+s2}{\PYZdq{}}\PY{p}{)}\PY{p}{)}\PY{p}{)}\PY{p}{,} \PY{n}{Node}\PY{p}{(}\PY{l+m+mi}{6}\PY{p}{,} \PY{l+m+mi}{234}\PY{p}{,} \PY{n}{Node}\PY{p}{(}\PY{l+m+mi}{5}\PY{p}{,} \PY{l+s+s2}{\PYZdq{}}\PY{l+s+s2}{abc}\PY{l+s+s2}{\PYZdq{}}\PY{p}{)}\PY{p}{,} \PY{n}{Node}\PY{p}{(}\PY{l+m+mi}{7}\PY{p}{,} \PY{l+s+s2}{\PYZdq{}}\PY{l+s+s2}{abv}\PY{l+s+s2}{\PYZdq{}}\PY{p}{,} \PY{n}{right}\PY{o}{=}\PY{n}{Node}\PY{p}{(}\PY{l+m+mi}{8}\PY{p}{,} \PY{l+s+s2}{\PYZdq{}}\PY{l+s+s2}{cde}\PY{l+s+s2}{\PYZdq{}}\PY{p}{)}\PY{p}{)}\PY{p}{)}\PY{p}{)}
\PY{n}{assert\PYZus{}equal}\PY{p}{(}\PY{p}{[}\PY{p}{(}\PY{l+m+mi}{1}\PY{p}{,} \PY{l+s+s1}{\PYZsq{}}\PY{l+s+s1}{asd}\PY{l+s+s1}{\PYZsq{}}\PY{p}{)}\PY{p}{,} \PY{p}{(}\PY{l+m+mi}{2}\PY{p}{,} \PY{l+m+mi}{123}\PY{p}{)}\PY{p}{,} \PY{p}{(}\PY{l+m+mi}{3}\PY{p}{,} \PY{k+kc}{None}\PY{p}{)}\PY{p}{,} \PY{p}{(}\PY{l+m+mi}{3}\PY{p}{,} \PY{l+s+s1}{\PYZsq{}}\PY{l+s+s1}{vwa}\PY{l+s+s1}{\PYZsq{}}\PY{p}{)}\PY{p}{,} \PY{p}{(}\PY{l+m+mi}{4}\PY{p}{,} \PY{l+m+mi}{421}\PY{p}{)}\PY{p}{,} \PY{p}{(}\PY{l+m+mi}{5}\PY{p}{,} \PY{l+s+s1}{\PYZsq{}}\PY{l+s+s1}{abc}\PY{l+s+s1}{\PYZsq{}}\PY{p}{)}\PY{p}{,} \PY{p}{(}\PY{l+m+mi}{6}\PY{p}{,} \PY{l+m+mi}{234}\PY{p}{)}\PY{p}{,} \PY{p}{(}\PY{l+m+mi}{7}\PY{p}{,} \PY{l+s+s1}{\PYZsq{}}\PY{l+s+s1}{abv}\PY{l+s+s1}{\PYZsq{}}\PY{p}{)}\PY{p}{,} \PY{p}{(}\PY{l+m+mi}{8}\PY{p}{,} \PY{l+s+s1}{\PYZsq{}}\PY{l+s+s1}{cde}\PY{l+s+s1}{\PYZsq{}}\PY{p}{)}\PY{p}{]}\PY{p}{,}
             \PY{n}{bin\PYZus{}tree\PYZus{}2\PYZus{}list}\PY{p}{(}\PY{n}{valid\PYZus{}bst}\PY{p}{)}\PY{p}{)}


\PY{n}{invalid\PYZus{}bst} \PY{o}{=} \PY{n}{Node}\PY{p}{(}\PY{l+m+mi}{1}\PY{p}{,} \PY{l+m+mi}{421}\PY{p}{,} \PY{n}{Node}\PY{p}{(}\PY{l+m+mi}{2}\PY{p}{,} \PY{l+s+s2}{\PYZdq{}}\PY{l+s+s2}{kve}\PY{l+s+s2}{\PYZdq{}}\PY{p}{,} \PY{n}{Node}\PY{p}{(}\PY{l+m+mi}{4}\PY{p}{,} \PY{l+s+s2}{\PYZdq{}}\PY{l+s+s2}{abc}\PY{l+s+s2}{\PYZdq{}}\PY{p}{,} \PY{n}{Node}\PY{p}{(}\PY{l+m+mi}{0}\PY{p}{,} \PY{l+s+s2}{\PYZdq{}}\PY{l+s+s2}{esi}\PY{l+s+s2}{\PYZdq{}}\PY{p}{)}\PY{p}{)}\PY{p}{)}\PY{p}{,} \PY{n}{Node}\PY{p}{(}\PY{l+m+mi}{9}\PY{p}{,} \PY{l+s+s2}{\PYZdq{}}\PY{l+s+s2}{agg}\PY{l+s+s2}{\PYZdq{}}\PY{p}{,} \PY{n}{right}\PY{o}{=}\PY{n}{Node}\PY{p}{(}\PY{l+m+mi}{4}\PY{p}{,} \PY{l+s+s2}{\PYZdq{}}\PY{l+s+s2}{dnf}\PY{l+s+s2}{\PYZdq{}}\PY{p}{,} \PY{n}{Node}\PY{p}{(}\PY{l+m+mi}{2}\PY{p}{,} \PY{l+m+mi}{123}\PY{p}{)}\PY{p}{)}\PY{p}{)}\PY{p}{)}
\PY{n}{assert\PYZus{}equal}\PY{p}{(}\PY{p}{[}\PY{p}{(}\PY{l+m+mi}{0}\PY{p}{,} \PY{l+s+s1}{\PYZsq{}}\PY{l+s+s1}{esi}\PY{l+s+s1}{\PYZsq{}}\PY{p}{)}\PY{p}{,} \PY{p}{(}\PY{l+m+mi}{4}\PY{p}{,} \PY{l+s+s1}{\PYZsq{}}\PY{l+s+s1}{abc}\PY{l+s+s1}{\PYZsq{}}\PY{p}{)}\PY{p}{,} \PY{p}{(}\PY{l+m+mi}{2}\PY{p}{,} \PY{l+s+s1}{\PYZsq{}}\PY{l+s+s1}{kve}\PY{l+s+s1}{\PYZsq{}}\PY{p}{)}\PY{p}{,} \PY{p}{(}\PY{l+m+mi}{1}\PY{p}{,} \PY{l+m+mi}{421}\PY{p}{)}\PY{p}{,} \PY{p}{(}\PY{l+m+mi}{9}\PY{p}{,} \PY{l+s+s1}{\PYZsq{}}\PY{l+s+s1}{agg}\PY{l+s+s1}{\PYZsq{}}\PY{p}{)}\PY{p}{,} \PY{p}{(}\PY{l+m+mi}{2}\PY{p}{,} \PY{l+m+mi}{123}\PY{p}{)}\PY{p}{,} \PY{p}{(}\PY{l+m+mi}{4}\PY{p}{,} \PY{l+s+s1}{\PYZsq{}}\PY{l+s+s1}{dnf}\PY{l+s+s1}{\PYZsq{}}\PY{p}{)}\PY{p}{]}\PY{p}{,}
             \PY{n}{bin\PYZus{}tree\PYZus{}2\PYZus{}list}\PY{p}{(}\PY{n}{invalid\PYZus{}bst}\PY{p}{)}\PY{p}{)}
\end{Verbatim}
\end{tcolorbox}

    \begin{tcolorbox}[breakable, size=fbox, boxrule=1pt, pad at break*=1mm,colback=cellbackground, colframe=cellborder]
\prompt{In}{incolor}{15}{\boxspacing}
\begin{Verbatim}[commandchars=\\\{\}]
\PY{c+c1}{\PYZsh{} hidden tests}
\end{Verbatim}
\end{tcolorbox}

    \hypertarget{d-list_2_bin_tree---3p.}{%
\subsection{d) list\_2\_bin\_tree() -
3P.}\label{d-list_2_bin_tree---3p.}}

Implementiere Sie die Funktion \texttt{list\_2\_bin\_tree()}, die als
Parameter einen BT \texttt{node} und eine Liste \texttt{node\_list}
übergegeben bekommt. Hierbei werden die Tupel aus Schlüssel und Wert aus
der Liste \texttt{node\_list} mit Inorder-Traversal in den BT
eingesetzt.

    \begin{tcolorbox}[breakable, size=fbox, boxrule=1pt, pad at break*=1mm,colback=cellbackground, colframe=cellborder]
\prompt{In}{incolor}{16}{\boxspacing}
\begin{Verbatim}[commandchars=\\\{\}]
\PY{k}{def} \PY{n+nf}{list\PYZus{}2\PYZus{}bin\PYZus{}tree}\PY{p}{(}\PY{n}{node}\PY{p}{,} \PY{n}{node\PYZus{}list}\PY{p}{)}\PY{p}{:}
    \PY{l+s+sd}{\PYZdq{}\PYZdq{}\PYZdq{} Convert keys \PYZam{} values from a list into a binary tree. \PYZdq{}\PYZdq{}\PYZdq{}}
    \PY{n}{curr} \PY{o}{=} \PY{n}{node}
    \PY{n}{in\PYZus{}order\PYZus{}stack} \PY{o}{=} \PY{p}{[}\PY{p}{]}
    \PY{n}{i} \PY{o}{=} \PY{l+m+mi}{0}
    \PY{k}{while} \PY{k+kc}{True}\PY{p}{:}
        \PY{k}{if} \PY{n}{curr} \PY{o+ow}{is} \PY{o+ow}{not} \PY{k+kc}{None}\PY{p}{:}
            \PY{n}{in\PYZus{}order\PYZus{}stack}\PY{o}{.}\PY{n}{append}\PY{p}{(}\PY{n}{curr}\PY{p}{)}
            \PY{n}{curr} \PY{o}{=} \PY{n}{curr}\PY{o}{.}\PY{n}{left}
        \PY{k}{elif}\PY{p}{(}\PY{n}{in\PYZus{}order\PYZus{}stack}\PY{p}{)}\PY{p}{:}
            \PY{n}{curr} \PY{o}{=} \PY{n}{in\PYZus{}order\PYZus{}stack}\PY{o}{.}\PY{n}{pop}\PY{p}{(}\PY{p}{)}
            \PY{n}{curr}\PY{o}{.}\PY{n}{key} \PY{o}{=} \PY{n}{node\PYZus{}list}\PY{p}{[}\PY{n}{i}\PY{p}{]}\PY{p}{[}\PY{l+m+mi}{0}\PY{p}{]}
            \PY{n}{curr}\PY{o}{.}\PY{n}{value} \PY{o}{=} \PY{n}{node\PYZus{}list}\PY{p}{[}\PY{n}{i}\PY{p}{]}\PY{p}{[}\PY{l+m+mi}{1}\PY{p}{]}
            \PY{n}{i} \PY{o}{+}\PY{o}{=} \PY{l+m+mi}{1}
            \PY{n}{curr} \PY{o}{=} \PY{n}{curr}\PY{o}{.}\PY{n}{right}
        \PY{k}{else}\PY{p}{:}
            \PY{k}{break}
\end{Verbatim}
\end{tcolorbox}

    \hypertarget{d-tests}{%
\subsection{d) Tests}\label{d-tests}}

    \begin{tcolorbox}[breakable, size=fbox, boxrule=1pt, pad at break*=1mm,colback=cellbackground, colframe=cellborder]
\prompt{In}{incolor}{17}{\boxspacing}
\begin{Verbatim}[commandchars=\\\{\}]
\PY{n}{tree} \PY{o}{=} \PY{n}{Node}\PY{p}{(}\PY{l+m+mi}{0}\PY{p}{,} \PY{l+m+mi}{0}\PY{p}{,} \PY{n}{Node}\PY{p}{(}\PY{l+m+mi}{1}\PY{p}{,} \PY{l+m+mi}{1}\PY{p}{,} \PY{n}{Node}\PY{p}{(}\PY{l+m+mi}{2}\PY{p}{,} \PY{l+m+mi}{2}\PY{p}{)}\PY{p}{,} \PY{n}{Node}\PY{p}{(}\PY{l+m+mi}{3}\PY{p}{,} \PY{l+m+mi}{3}\PY{p}{)}\PY{p}{)}\PY{p}{,} \PY{n}{Node}\PY{p}{(}\PY{l+m+mi}{4}\PY{p}{,} \PY{l+m+mi}{4}\PY{p}{,} \PY{n}{Node}\PY{p}{(}\PY{l+m+mi}{5}\PY{p}{,} \PY{l+m+mi}{5}\PY{p}{)}\PY{p}{,} \PY{n}{Node}\PY{p}{(}\PY{l+m+mi}{6}\PY{p}{,} \PY{l+m+mi}{6}\PY{p}{,} \PY{n}{Node}\PY{p}{(}\PY{l+m+mi}{7}\PY{p}{,} \PY{l+m+mi}{7}\PY{p}{)}\PY{p}{)}\PY{p}{)}\PY{p}{)}
\PY{n+nb}{list} \PY{o}{=} \PY{p}{[}\PY{p}{(}\PY{l+m+mi}{1}\PY{p}{,} \PY{l+s+s1}{\PYZsq{}}\PY{l+s+s1}{asd}\PY{l+s+s1}{\PYZsq{}}\PY{p}{)}\PY{p}{,} \PY{p}{(}\PY{l+m+mi}{2}\PY{p}{,} \PY{l+m+mi}{123}\PY{p}{)}\PY{p}{,} \PY{p}{(}\PY{l+m+mi}{3}\PY{p}{,} \PY{k+kc}{None}\PY{p}{)}\PY{p}{,} \PY{p}{(}\PY{l+m+mi}{3}\PY{p}{,} \PY{l+s+s1}{\PYZsq{}}\PY{l+s+s1}{vwa}\PY{l+s+s1}{\PYZsq{}}\PY{p}{)}\PY{p}{,} \PY{p}{(}\PY{l+m+mi}{4}\PY{p}{,} \PY{l+m+mi}{421}\PY{p}{)}\PY{p}{,} \PY{p}{(}\PY{l+m+mi}{5}\PY{p}{,} \PY{l+s+s1}{\PYZsq{}}\PY{l+s+s1}{abc}\PY{l+s+s1}{\PYZsq{}}\PY{p}{)}\PY{p}{,} \PY{p}{(}\PY{l+m+mi}{6}\PY{p}{,} \PY{l+m+mi}{234}\PY{p}{)}\PY{p}{,} \PY{p}{(}\PY{l+m+mi}{7}\PY{p}{,} \PY{l+s+s1}{\PYZsq{}}\PY{l+s+s1}{abv}\PY{l+s+s1}{\PYZsq{}}\PY{p}{)}\PY{p}{,} \PY{p}{(}\PY{l+m+mi}{8}\PY{p}{,} \PY{l+s+s1}{\PYZsq{}}\PY{l+s+s1}{cde}\PY{l+s+s1}{\PYZsq{}}\PY{p}{)}\PY{p}{]}
\PY{n}{list\PYZus{}2\PYZus{}bin\PYZus{}tree}\PY{p}{(}\PY{n}{tree}\PY{p}{,} \PY{n+nb}{list}\PY{p}{)}
\PY{n}{assert\PYZus{}equal}\PY{p}{(}\PY{n+nb}{str}\PY{p}{(}\PY{n}{tree}\PY{p}{)}\PY{p}{,} \PY{l+s+s2}{\PYZdq{}}\PY{l+s+s2}{(((None, 1|asd, None), 2|123, (None, 3|None, None)), 3|vwa, ((None, 4|421, None), 5|abc, ((None, 6|234, None), 7|abv, None)))}\PY{l+s+s2}{\PYZdq{}}\PY{p}{)}

\PY{n}{tree} \PY{o}{=} \PY{n}{Node}\PY{p}{(}\PY{l+m+mi}{0}\PY{p}{,} \PY{l+m+mi}{0}\PY{p}{,} \PY{n}{Node}\PY{p}{(}\PY{l+m+mi}{1}\PY{p}{,} \PY{l+m+mi}{1}\PY{p}{,} \PY{n}{Node}\PY{p}{(}\PY{l+m+mi}{2}\PY{p}{,} \PY{l+m+mi}{2}\PY{p}{,} \PY{n}{Node}\PY{p}{(}\PY{l+m+mi}{3}\PY{p}{,} \PY{l+m+mi}{3}\PY{p}{,} \PY{n}{Node}\PY{p}{(}\PY{l+m+mi}{4}\PY{p}{,} \PY{l+m+mi}{4}\PY{p}{,} \PY{n}{Node}\PY{p}{(}\PY{l+m+mi}{5}\PY{p}{,} \PY{l+m+mi}{5}\PY{p}{,} \PY{n}{Node}\PY{p}{(}\PY{l+m+mi}{6}\PY{p}{,} \PY{l+m+mi}{6}\PY{p}{)}\PY{p}{)}\PY{p}{)}\PY{p}{)}\PY{p}{)}\PY{p}{)}\PY{p}{)}
\PY{n+nb}{list} \PY{o}{=} \PY{p}{[}\PY{p}{(}\PY{l+m+mi}{0}\PY{p}{,} \PY{l+s+s1}{\PYZsq{}}\PY{l+s+s1}{esi}\PY{l+s+s1}{\PYZsq{}}\PY{p}{)}\PY{p}{,} \PY{p}{(}\PY{l+m+mi}{4}\PY{p}{,} \PY{l+s+s1}{\PYZsq{}}\PY{l+s+s1}{abc}\PY{l+s+s1}{\PYZsq{}}\PY{p}{)}\PY{p}{,} \PY{p}{(}\PY{l+m+mi}{2}\PY{p}{,} \PY{l+s+s1}{\PYZsq{}}\PY{l+s+s1}{kve}\PY{l+s+s1}{\PYZsq{}}\PY{p}{)}\PY{p}{,} \PY{p}{(}\PY{l+m+mi}{1}\PY{p}{,} \PY{l+m+mi}{421}\PY{p}{)}\PY{p}{,} \PY{p}{(}\PY{l+m+mi}{9}\PY{p}{,} \PY{l+s+s1}{\PYZsq{}}\PY{l+s+s1}{agg}\PY{l+s+s1}{\PYZsq{}}\PY{p}{)}\PY{p}{,} \PY{p}{(}\PY{l+m+mi}{2}\PY{p}{,} \PY{l+m+mi}{123}\PY{p}{)}\PY{p}{,} \PY{p}{(}\PY{l+m+mi}{4}\PY{p}{,} \PY{l+s+s1}{\PYZsq{}}\PY{l+s+s1}{dnf}\PY{l+s+s1}{\PYZsq{}}\PY{p}{)}\PY{p}{]}
\PY{n}{list\PYZus{}2\PYZus{}bin\PYZus{}tree}\PY{p}{(}\PY{n}{tree}\PY{p}{,} \PY{n+nb}{list}\PY{p}{)}
\PY{n}{assert\PYZus{}equal}\PY{p}{(}\PY{n+nb}{str}\PY{p}{(}\PY{n}{tree}\PY{p}{)}\PY{p}{,} \PY{l+s+s2}{\PYZdq{}}\PY{l+s+s2}{(((((((None, 0|esi, None), 4|abc, None), 2|kve, None), 1|421, None), 9|agg, None), 2|123, None), 4|dnf, None)}\PY{l+s+s2}{\PYZdq{}}\PY{p}{)}
\end{Verbatim}
\end{tcolorbox}

    \hypertarget{e-bin_tree_2_bin_search_tree---4p.}{%
\subsection{e) bin\_tree\_2\_bin\_search\_tree() -
4P.}\label{e-bin_tree_2_bin_search_tree---4p.}}

Implementieren Sie die Funktion
\texttt{bin\_tree\_2\_bin\_search\_tree()}, die überprüft ob ein
übergebener BT \texttt{node} auch ein BST ist und falls dies nicht der
Fall ist, diesen in einen BST überführt. Hierbei soll die ursprüngliche
Form (Verzweigungen) des BT erhalten bleiben. Nutzen Sie bei der Lösung
dieser Aufgabe, die von Ihnen zuvor implementierten Funktionen. Sie
können davon ausgehen, dass wir als Eingabe nur valide BT verwenden, die
z.B. keine doppelten Nodes besitzen.

Hinweis: Da der Fokus dieser Aufgabe nicht auf den Sortieralgorithmen
selbst liegt, dürfen Sie Funktionen wie \texttt{sort()} oder
\texttt{sorted()} in Ihrer Lösung verwenden.

    \begin{tcolorbox}[breakable, size=fbox, boxrule=1pt, pad at break*=1mm,colback=cellbackground, colframe=cellborder]
\prompt{In}{incolor}{18}{\boxspacing}
\begin{Verbatim}[commandchars=\\\{\}]
\PY{k}{def} \PY{n+nf}{bin\PYZus{}tree\PYZus{}2\PYZus{}bin\PYZus{}search\PYZus{}tree}\PY{p}{(}\PY{n}{node}\PY{p}{)}\PY{p}{:}
    \PY{l+s+sd}{\PYZdq{}\PYZdq{}\PYZdq{} Convert a binary tree into a binary search tree. Preserves the original structure of the binary tree. \PYZdq{}\PYZdq{}\PYZdq{}}
    \PY{k}{if} \PY{o+ow}{not} \PY{n}{is\PYZus{}bin\PYZus{}search\PYZus{}tree}\PY{p}{(}\PY{n}{node}\PY{p}{)}\PY{p}{:}
        \PY{n}{nodelist} \PY{o}{=} \PY{n}{bin\PYZus{}tree\PYZus{}2\PYZus{}list}\PY{p}{(}\PY{n}{node}\PY{p}{)}
        \PY{n}{nodelist}\PY{o}{.}\PY{n}{sort}\PY{p}{(}\PY{p}{)}
        \PY{n}{list\PYZus{}2\PYZus{}bin\PYZus{}tree}\PY{p}{(}\PY{n}{node}\PY{p}{,} \PY{n}{nodelist}\PY{p}{)}
\end{Verbatim}
\end{tcolorbox}

    \hypertarget{e-tests}{%
\subsection{e) Tests}\label{e-tests}}

    \begin{tcolorbox}[breakable, size=fbox, boxrule=1pt, pad at break*=1mm,colback=cellbackground, colframe=cellborder]
\prompt{In}{incolor}{19}{\boxspacing}
\begin{Verbatim}[commandchars=\\\{\}]
\PY{n}{invalid\PYZus{}bst} \PY{o}{=} \PY{n}{Node}\PY{p}{(}\PY{l+m+mi}{1}\PY{p}{,} \PY{l+m+mi}{421}\PY{p}{,} \PY{n}{Node}\PY{p}{(}\PY{l+m+mi}{2}\PY{p}{,} \PY{l+s+s2}{\PYZdq{}}\PY{l+s+s2}{kve}\PY{l+s+s2}{\PYZdq{}}\PY{p}{,} \PY{n}{Node}\PY{p}{(}\PY{l+m+mi}{4}\PY{p}{,} \PY{l+s+s2}{\PYZdq{}}\PY{l+s+s2}{abc}\PY{l+s+s2}{\PYZdq{}}\PY{p}{,} \PY{n}{Node}\PY{p}{(}\PY{l+m+mi}{0}\PY{p}{,} \PY{l+s+s2}{\PYZdq{}}\PY{l+s+s2}{esi}\PY{l+s+s2}{\PYZdq{}}\PY{p}{)}\PY{p}{)}\PY{p}{)}\PY{p}{,} \PY{n}{Node}\PY{p}{(}\PY{l+m+mi}{9}\PY{p}{,} \PY{l+s+s2}{\PYZdq{}}\PY{l+s+s2}{agg}\PY{l+s+s2}{\PYZdq{}}\PY{p}{,} \PY{n}{right}\PY{o}{=}\PY{n}{Node}\PY{p}{(}\PY{l+m+mi}{7}\PY{p}{,} \PY{l+s+s2}{\PYZdq{}}\PY{l+s+s2}{dnf}\PY{l+s+s2}{\PYZdq{}}\PY{p}{,} \PY{n}{Node}\PY{p}{(}\PY{l+m+mi}{3}\PY{p}{,} \PY{l+m+mi}{123}\PY{p}{)}\PY{p}{)}\PY{p}{)}\PY{p}{)}
\PY{n}{bin\PYZus{}tree\PYZus{}2\PYZus{}bin\PYZus{}search\PYZus{}tree}\PY{p}{(}\PY{n}{invalid\PYZus{}bst}\PY{p}{)}
\PY{n}{assert\PYZus{}equal}\PY{p}{(}\PY{k+kc}{True}\PY{p}{,} \PY{n}{is\PYZus{}bin\PYZus{}search\PYZus{}tree}\PY{p}{(}\PY{p}{(}\PY{n}{invalid\PYZus{}bst}\PY{p}{)}\PY{p}{)}\PY{p}{)}
\end{Verbatim}
\end{tcolorbox}

    \begin{tcolorbox}[breakable, size=fbox, boxrule=1pt, pad at break*=1mm,colback=cellbackground, colframe=cellborder]
\prompt{In}{incolor}{20}{\boxspacing}
\begin{Verbatim}[commandchars=\\\{\}]
\PY{c+c1}{\PYZsh{} hidden tests}
\end{Verbatim}
\end{tcolorbox}

    \hypertarget{jupyter-notebook-stolperfalle}{%
\subsection{Jupyter Notebook
Stolperfalle}\label{jupyter-notebook-stolperfalle}}

Bei der Benutzung von Jupyter Notebooks, wird der globale Zustand aller
Variablen zwischen der Ausführung von verschiedenen Zellen erhalten.
Dies ist auch der Fall, wenn Zellen gelöscht oder hinzugefügt werden. Um
sicher zu gehen, dass nicht ausversehen notwendige Variablen
überschrieben oder gelöscht wurden, kann der Befehl
\texttt{Kernel\ -\textgreater{}\ Restart\ \&\ Run\ All} ausgeführt
werden.


    % Add a bibliography block to the postdoc
    
    
    
\end{document}
